% !TeX encoding = UTF-8
% !TeX program = xelatex
% !TeX spellcheck = en_US

\documentclass[degree=master]{thuthesis}
  % 学位 degree:
  %   doctor | master | bachelor | postdoc
  % 学位类型 degree-type:
  %   academic(默认)| professional
  % 语言 language
  %   chinese(默认)| english
  % 字体库 fontset
  %   windows | mac | fandol | ubuntu
  % 建议终版使用 Windows 平台的字体编译


% 论文基本配置,加载宏包等全局配置
% !TeX root = ./thuthesis-example.tex

% 论文基本信息配置

\thusetup{
  %******************************
  % 注意:
  %   1. 配置里面不要出现空行
  %   2. 不需要的配置信息可以删除
  %   3. 建议先阅读文档中所有关于选项的说明
  %******************************
  %
  % 输出格式
  %   选择打印版(print)或用于提交的电子版(electronic),前者会插入空白页以便直接双面打印
  %
  output = print,
  % 格式类型
  %   默认为论文(thesis),也可以设置为开题报告(proposal)
  % thesis-type = proposal,
  %
  % 标题
  %   可使用“\\”命令手动控制换行
  %
  title  = {量子线路的观测值模拟},
  title* = {Simulating observables of quantum circuits},
  %
  % 学科门类
  %   1. 学术型
  %      - 中文
  %        需注明所属的学科门类,例如:
  %        哲学、经济学、法学、教育学、文学、历史学、理学、工学、农学、医学、
  %        军事学、管理学、艺术学
  %      - 英文
  %        博士:Doctor of Philosophy
  %        硕士:
  %          哲学、文学、历史学、法学、教育学、艺术学门类,公共管理学科
  %          填写“Master of Arts“,其它填写“Master of Science”
  %   2. 专业型
  %      直接填写专业学位的名称,例如:
  %      教育博士、工程硕士等
  %      Doctor of Education, Master of Engineering
  %   3. 本科生不需要填写
  %
  degree-category  = {理学博士},
  degree-category* = {Doctor of Philosophy},
  %
  % 培养单位
  %   填写所属院系的全名
  %
  department = {数学科学系},
  %
  % 学科
  %   1. 研究生学术型学位,获得一级学科授权的学科填写一级学科名称,其他填写二级学科名称
  %   2. 本科生填写专业名称,第二学位论文需标注“(第二学位)”
  %
  discipline  = {数学},
  discipline* = {Mathmathics},
  %
  % 专业领域
  %   1. 设置专业领域的专业学位类别,填写相应专业领域名称
  %   2. 2019 级及之前工程硕士学位论文,在 `engineering-field` 填写相应工程领域名称
  %   3. 其他专业学位类别的学位论文无需此信息
  %
  % professional-field  = {计算机技术},
  % professional-field* = {Computer Technology},
  %
  % 姓名
  %
  author  = {邵钰菓},
  author* = {Yuguo Shao},
  %
  % 学号
  % 仅当书写开题报告时需要(同时设置 `thesis-type = proposal')
  %
  % student-id = {2000310000},
  %
  % 指导教师
  %   中文姓名和职称之间以英文逗号“,”分开,下同
  %
  supervisor  = {刘正伟, 教授},
  supervisor* = {Professor Zhengwei Liu},
  %
  % 副指导教师
  %
  %associate-supervisor  = {陈文光, 教授},
  %associate-supervisor* = {Professor Chen Wenguang},
  %
  % 联合指导教师
  %
  % co-supervisor  = {某某某, 教授},
  % co-supervisor* = {Professor Mou Moumou},
  %
  % 日期
  %   使用 ISO 格式;默认为当前时间
  %
  % date = {2019-07-07},
  %
  % 是否在中文封面后的空白页生成书脊(默认 false)
  %
  include-spine = false,
  %
  % 密级和年限
  %   秘密, 机密, 绝密
  %
  % secret-level = {秘密},
  % secret-year  = {10},
  %
  % 博士后专有部分
  %
  % clc                = {分类号},
  % udc                = {UDC},
  % id                 = {编号},
  % discipline-level-1 = {计算机科学与技术},  % 流动站(一级学科)名称
  % discipline-level-2 = {系统结构},          % 专业(二级学科)名称
  % start-date         = {2011-07-01},        % 研究工作起始时间
}

% 载入所需的宏包

% 定理类环境宏包
\usepackage{amsthm}
% 也可以使用 ntheorem
% \usepackage[amsmath,thmmarks,hyperref]{ntheorem}

\thusetup{
  %
  % 数学字体
  % math-style = GB,  % GB | ISO | TeX
  math-font  = xits,  % stix | xits | libertinus
}

% 可以使用 nomencl 生成符号和缩略语说明
% \usepackage{nomencl}
% \makenomenclature

% 表格加脚注
\usepackage{threeparttable}

% 表格中支持跨行
\usepackage{multirow}
\usepackage{physics}
% 固定宽度的表格。
% \usepackage{tabularx}

% 跨页表格
\usepackage{longtable}

% 算法
\usepackage{algorithm}
\usepackage{algorithmic}

% 量和单位
\usepackage{siunitx}
\usepackage{quantikz}

% 参考文献使用 BibTeX + natbib 宏包
% 顺序编码制
\usepackage[sort]{natbib}
\bibliographystyle{thuthesis-numeric}

% 著者-出版年制
% \usepackage{natbib}
% \bibliographystyle{thuthesis-author-year}

% 生命科学学院要求使用 Cell 参考文献格式(2023 年以前使用 author-date 格式)
% \usepackage{natbib}
% \bibliographystyle{cell}

% 本科生参考文献的著录格式
% \usepackage[sort]{natbib}
% \bibliographystyle{thuthesis-bachelor}

% 参考文献使用 BibLaTeX 宏包
% \usepackage[style=thuthesis-numeric]{biblatex}
% \usepackage[style=thuthesis-author-year]{biblatex}
% \usepackage[style=gb7714-2015]{biblatex}
% \usepackage[style=apa]{biblatex}
% \usepackage[style=mla-new]{biblatex}
% 声明 BibLaTeX 的数据库
% \addbibresource{ref/refs.bib}

% 定义所有的图片文件在 figures 子目录下
\graphicspath{{figures/}}

% 数学命令
\makeatletter
\newcommand\dif{%  % 微分符号
  \mathop{}\!%
  \ifthu@math@style@TeX
    d%
  \else
    \mathrm{d}%
  \fi
}
\makeatother

% hyperref 宏包在最后调用
\usepackage{hyperref}

\newcommand{\supp}{\text{supp}}
\newcommand{\poly}{\text{Poly}}


\begin{document}

% 封面
\maketitle

% 学位论文指导小组、公开评阅人和答辩委员会名单
% 本科生不需要
% !TeX root = ../thuthesis-example.tex

%\begin{committee}[name={学位论文指导小组、公开评阅人和答辩委员会名单}]
\begin{committee}[name={学位论文公开评阅人和答辩委员会名单}]
  \newcolumntype{C}[1]{@{}>{\centering\arraybackslash}p{#1}}

  \section*{指导小组名单}

  \begin{center}
    \begin{tabular}{C{3cm}C{3cm}C{9cm}@{}}
      刘正伟 & 教授     & 清华大学 
  %    王XX & 副教授   & 清华大学 \\
  %    张XX & 助理教授 & 清华大学 \\
    \end{tabular}
  \end{center}


  \section*{公开评阅人名单}

  \begin{center}
    \begin{tabular}{C{3cm}C{3cm}C{9cm}@{}}
      骆顺龙	& 研究员   & 中国科学院数学与系统科学研究院\\
      魏朝晖 & 助理教授 & 清华大学\\
      张潘 & 研究员 & 中国科学院理论物理研究所 \\
      袁骁 & 助理教授 & 北京大学 
    \end{tabular}
  \end{center}


  \section*{答辩委员会名单}

  \begin{center}
    \begin{tabular}{C{2.75cm}C{2.98cm}C{4.63cm}C{4.63cm}@{}}
      主席 & 郑浩                  & 教授                    & 清华大学       \\
      委员 & 肖杰                  & 教授                    & 北京师范大学       \\
         & \multirow{2}{*}{骆顺龙} & \multirow{2}{*}{研究员} & 中国科学院 \\
         &                       &                         & 数学与系统科学研究院  \\
        %& 骆顺龙                  & 研究员                    & 中国科学院数学与系统科学研究院       \\
          & 左怀青                  & 教授                  & 清华大学       \\
          & 魏朝晖                  & 助理教授                  & 清华大学       \\
      秘书 & 李慧慧                  & 博士后              & 清华大学       \\
    \end{tabular}
  \end{center}
\end{committee}



% 也可以导入 Word 版转的 PDF 文件
% \begin{committee}[file=figures/committee.pdf]
% \end{committee}


% 使用授权的说明
% 本科生开题报告不需要
\copyrightpage
% 将签字扫描后授权文件 scan-copyright.pdf 替换原始页面
% \copyrightpage[file=scan-copyright.pdf]

\frontmatter
% !TeX root = ../thuthesis-example.tex

% 中英文摘要和关键字

\begin{abstract}
  量子计算的经典模拟是验证量子硬件、优化算法设计及探索量子优势边界的重要工具。本文针对含噪声中等规模量子(NISQ)设备的模拟需求,提出基于Pauli路径积分的高效经典模拟方法,系统分析其计算复杂度与误差特性,并验证其实际应用价值。

  大规模变分量子算法被广泛认为是实现实用量子优势的潜在途径。然而,量子噪声的存在可能会抑制并削弱这些优势,这使得经典可模拟性的界限变得模糊。为了进一步澄清这一问题,我们提出了一种基于可观测量在泡利路径上反向传播路径积分(OBPPP)的新型多项式规模方法。该方法能够在单量子比特泡利噪声存在的情况下,以有界截断误差高效近似变分量子算法中算子的期望值。理论上我们严格证明:1)当最小非零噪声率γ为常数时,OBPPP的时间与空间复杂度与量子比特数n、电路深度L呈现多项式关系;2)对于可变的γ,当存在超过两个非零噪声因子时,若γ大于1/log L则复杂度保持Poly(n, L),但若γ低于1/L时复杂度将随L呈指数增长。数值方面,我们对IBM在127量子比特鹰处理器上进行的零噪声外推实验结果[Nature 618, 500 (2023)]进行了经典模拟。相比量子设备,我们的方法获得了更高的精度与更快的运行速度。此外,我们的方法能够模拟含噪声结果,精确复现了IBM未经误差缓解、与原始实验观测直接对应的计算结果。本研究表明噪声在经典模拟中的关键作用,所提出的方法适用于广泛量子电路的期望值计算,在量子计算机验证领域具有普适性。



  论文的摘要是对论文研究内容和成果的高度概括。
  摘要应对论文所研究的问题及其研究目的进行描述,对研究方法和过程进行简单介绍,对研究成果和所得结论进行概括。
  摘要应具有独立性和自明性,其内容应包含与论文全文同等量的主要信息。
  使读者即使不阅读全文,通过摘要就能了解论文的总体内容和主要成果。

  论文摘要的书写应力求精确、简明。
  切忌写成对论文书写内容进行提要的形式,尤其要避免“第 1 章……;第 2 章……;……”这种或类似的陈述方式。

  关键词是为了文献标引工作、用以表示全文主要内容信息的单词或术语。
  关键词不超过 5 个,每个关键词中间用分号分隔。

  % 关键词用“英文逗号”分隔,输出时会自动处理为正确的分隔符
  \thusetup{
    keywords = {量子计算, 量子线路经典模拟 , Pauli路径积分, 变分量子算法, 量子噪声},
  }
\end{abstract}

\begin{abstract*}
  An abstract of a dissertation is a summary and extraction of research work and contributions.
  Included in an abstract should be description of research topic and research objective, brief introduction to methodology and research process, and summary of conclusion and contributions of the research.
  An abstract should be characterized by independence and clarity and carry identical information with the dissertation.
  It should be such that the general idea and major contributions of the dissertation are conveyed without reading the dissertation.

  An abstract should be concise and to the point.
  It is a misunderstanding to make an abstract an outline of the dissertation and words “the first chapter”, “the second chapter” and the like should be avoided in the abstract.

  Keywords are terms used in a dissertation for indexing, reflecting core information of the dissertation.
  An abstract may contain a maximum of 5 keywords, with semi-colons used in between to separate one another.

  % Use comma as separator when inputting
  \thusetup{
    keywords* = {keyword 1, keyword 2, keyword 3, keyword 4, keyword 5},
  }
\end{abstract*}


% 目录
\tableofcontents

% 插图和附表清单
% 本科生的插图索引和表格索引需要移至正文之后、参考文献前
% \listoffiguresandtables  % 插图和附表清单(仅限研究生)
\listoffigures           % 插图清单
\listoftables            % 附表清单

% 符号对照表
% !TeX root = ../thuthesis-example.tex

\begin{denotation}[3cm]
  \item[CPD] Clifford数据回归(Clifford Data Regression)
  \item[CPDR] Clifford扰动数据回归(Clifford Perturbation Data Regression)
  \item[CPTP] 完全正保迹(Completely Positive and Trace-Preserving) 
  \item[DMRG] 密度矩阵重整化群(Density Matrix Renormalization Group)
  \item[MPS] 矩阵乘积态(Matrix Product State) 
  \item[MSE] 均方误差(Mean Squared Error) 
  \item[NISQ] 含噪声中等规模量子(Noisy Intermediate-Scale Quantum)
  \item[OBPPP] 可观测量在Pauli路径下的反向传播算法(Observable's Back-Propagation on Pauli Path)
  \item[POVM] 正算子值测量(Positive Operator-Valued Measure)
  \item[QAOA] 量子近似优化算法(Quantum Approximate Optimization Algorithm)
  \item[QEC] 量子纠错(Quantum Error Correction)
  \item[Qubit] 量子比特(Quantum Bit)
  \item[QSVM] 量子支持向量机(Quantum Support Vector Machine) 
  \item[QNN] 量子神经网络(Quantum Neural Network) 
  \item[RRMSE] 相对均方根误差(Relative Root Mean Square Error) 
  \item[SPD] 稀疏Pauli动力学 (Sparse Pauli Dynamics)
  \item[SVD] 奇异值分解(Singular Value Decomposition)
  \item[VQA] 变分量子算法(Variational Quantum Algorithm) 
  \item[VQE] 变分量子本征求解器(Variational Quantum Eigensolver)
  \item[ZNE] 零噪声推断(Zero-Noise Extrapolation)
  
  
\end{denotation}



% 也可以使用 nomencl 宏包,需要在导言区
% \usepackage{nomencl}
% \makenomenclature

% 在这里输出符号说明
% \printnomenclature[3cm]

% 在正文中的任意为都可以标题
% \nomenclature{PI}{聚酰亚胺}
% \nomenclature{MPI}{聚酰亚胺模型化合物,N-苯基邻苯酰亚胺}
% \nomenclature{PBI}{聚苯并咪唑}
% \nomenclature{MPBI}{聚苯并咪唑模型化合物,N-苯基苯并咪唑}
% \nomenclature{PY}{聚吡咙}
% \nomenclature{PMDA-BDA}{均苯四酸二酐与联苯四胺合成的聚吡咙薄膜}
% \nomenclature{MPY}{聚吡咙模型化合物}
% \nomenclature{As-PPT}{聚苯基不对称三嗪}
% \nomenclature{MAsPPT}{聚苯基不对称三嗪单模型化合物,3,5,6-三苯基-1,2,4-三嗪}
% \nomenclature{DMAsPPT}{聚苯基不对称三嗪双模型化合物(水解实验模型化合物)}
% \nomenclature{S-PPT}{聚苯基对称三嗪}
% \nomenclature{MSPPT}{聚苯基对称三嗪模型化合物,2,4,6-三苯基-1,3,5-三嗪}
% \nomenclature{PPQ}{聚苯基喹噁啉}
% \nomenclature{MPPQ}{聚苯基喹噁啉模型化合物,3,4-二苯基苯并二嗪}
% \nomenclature{HMPI}{聚酰亚胺模型化合物的质子化产物}
% \nomenclature{HMPY}{聚吡咙模型化合物的质子化产物}
% \nomenclature{HMPBI}{聚苯并咪唑模型化合物的质子化产物}
% \nomenclature{HMAsPPT}{聚苯基不对称三嗪模型化合物的质子化产物}
% \nomenclature{HMSPPT}{聚苯基对称三嗪模型化合物的质子化产物}
% \nomenclature{HMPPQ}{聚苯基喹噁啉模型化合物的质子化产物}
% \nomenclature{PDT}{热分解温度}
% \nomenclature{HPLC}{高效液相色谱(High Performance Liquid Chromatography)}
% \nomenclature{HPCE}{高效毛细管电泳色谱(High Performance Capillary lectrophoresis)}
% \nomenclature{LC-MS}{液相色谱-质谱联用(Liquid chromatography-Mass Spectrum)}
% \nomenclature{TIC}{总离子浓度(Total Ion Content)}
% \nomenclature{\textit{ab initio}}{基于第一原理的量子化学计算方法,常称从头算法}
% \nomenclature{DFT}{密度泛函理论(Density Functional Theory)}
% \nomenclature{$E_a$}{化学反应的活化能(Activation Energy)}
% \nomenclature{ZPE}{零点振动能(Zero Vibration Energy)}
% \nomenclature{PES}{势能面(Potential Energy Surface)}
% \nomenclature{TS}{过渡态(Transition State)}
% \nomenclature{TST}{过渡态理论(Transition State Theory)}
% \nomenclature{$\increment G^\neq$}{活化自由能(Activation Free Energy)}
% \nomenclature{$\kappa$}{传输系数(Transmission Coefficient)}
% \nomenclature{IRC}{内禀反应坐标(Intrinsic Reaction Coordinates)}
% \nomenclature{$\nu_i$}{虚频(Imaginary Frequency)}
% \nomenclature{ONIOM}{分层算法(Our own N-layered Integrated molecular Orbital and molecular Mechanics)}
% \nomenclature{SCF}{自洽场(Self-Consistent Field)}
% \nomenclature{SCRF}{自洽反应场(Self-Consistent Reaction Field)}



% 正文部分
\mainmatter
\input{data/chap01}
% !TeX root = ../thuthesis-example.tex

\chapter{图表示例}

\section{插图}

图片通常在 \env{figure} 环境中使用 \cs{includegraphics} 插入,如图~\ref{fig:example} 的源代码。
建议矢量图片使用 PDF 格式,比如数据可视化的绘图;
照片应使用 JPG 格式;
其他的栅格图应使用无损的 PNG 格式。
注意,LaTeX 不支持 TIFF 格式;EPS 格式已经过时。

\begin{figure}
  \centering
  \includegraphics[width=0.5\linewidth]{example-image-a.pdf}
  \caption*{国外的期刊习惯将图表的标题和说明文字写成一段,需要改写为标题只含图表的名称,其他说明文字以注释方式写在图表下方,或者写在正文中。}
  \caption{示例图片标题}
  \label{fig:example}
\end{figure}

若图或表中有附注,采用英文小写字母顺序编号,附注写在图或表的下方。
国外的期刊习惯将图表的标题和说明文字写成一段,需要改写为标题只含图表的名称,其他说明文字以注释方式写在图表下方,或者写在正文中。

如果一个图由两个或两个以上分图组成时,各分图分别以 (a)、(b)、(c)...... 作为图序,并须有分图题。
推荐使用 \pkg{subcaption} 宏包来处理, 比如图~\ref{fig:subfig-a} 和图~\ref{fig:subfig-b}。

\begin{figure}
  \centering
  \subcaptionbox{分图 A\label{fig:subfig-a}}
    {\includegraphics[width=0.35\linewidth]{example-image-a.pdf}}
  \subcaptionbox{分图 B\label{fig:subfig-b}}
    {\includegraphics[width=0.35\linewidth]{example-image-b.pdf}}
  \caption{多个分图的示例}
  \label{fig:multi-image}
\end{figure}



\section{表格}

表应具有自明性。表中参数应标明量和单位的符号。
为使表格简洁易读,均采用三线表(例如表~\ref{tab:three-line})。
必要时可加辅助线,三线表无法清晰表达时可采用其他格式。

表序与表题置于表的上方。表单元格中的文字一般应居中书写(上下居中,左右居中),
不宜左右居中书写的,可采取两端对齐的方式书写。

\begin{table}
  \centering
  \caption{三线表示例}
  \begin{tabular}{cc}
    \toprule
    文件名          & 描述                         \\
    \midrule
    thuthesis.dtx   & 模板的源文件,包括文档和注释 \\
    thuthesis.cls   & 模板文件                     \\
    thuthesis-*.bst & BibTeX 参考文献表样式文件    \\
    \bottomrule
  \end{tabular}
  \label{tab:three-line}
\end{table}

若表中有附注,采用英文小写字母顺序编号,附注写在表的下方。
推荐使用 \pkg{threeparttable} 宏包。

\begin{table}
  \centering
  \begin{threeparttable}[c]
    \caption{带附注的表格示例}
    \label{tab:three-part-table}
    \begin{tabular}{cc}
      \toprule
      文件名                 & 描述                         \\
      \midrule
      thuthesis.dtx\tnote{a} & 模板的源文件,包括文档和注释 \\
      thuthesis.cls\tnote{b} & 模板文件                     \\
      thuthesis-*.bst        & BibTeX 参考文献表样式文件    \\
      \bottomrule
    \end{tabular}
    \begin{tablenotes}
      \item [a] 可以通过 xelatex 编译生成模板的使用说明文档;
        使用 xetex 编译 \file{thuthesis.ins} 时则会从 \file{.dtx} 中去除掉文档和注释,得到精简的 \file{.cls} 文件。
      \item [b] 更新模板时,一定要记得编译生成 \file{.cls} 文件,否则编译论文时载入的依然是旧版的模板。
    \end{tablenotes}
  \end{threeparttable}
\end{table}

如某个表需要转页接排,可以“续表”的形式另页打印,格式同前,只需在每页表序前加“续”字即可。
续表均应重复表头。
推荐使用 \pkg{longtable} 宏包。

\begin{longtable}{cccc}
    \caption{跨页长表格的表题}
    \label{tab:longtable} \\
    \toprule
    表头 1 & 表头 2 & 表头 3 & 表头 4 \\
    \midrule
  \endfirsthead
    \caption*{续表~\thetable\quad 跨页长表格的表题} \\
    \toprule
    表头 1 & 表头 2 & 表头 3 & 表头 4 \\
    \midrule
  \endhead
    \bottomrule
  \endfoot
  Row 1  & & & \\
  Row 2  & & & \\
  Row 3  & & & \\
  Row 4  & & & \\
  Row 5  & & & \\
  Row 6  & & & \\
  Row 7  & & & \\
  Row 8  & & & \\
  Row 9  & & & \\
  Row 10 & & & \\
\end{longtable}



\section{算法}

算法环境可以使用 \pkg{algorithms} 或者 \pkg{algorithm2e} 宏包。

\renewcommand{\algorithmicrequire}{\textbf{输入:}\unskip}
\renewcommand{\algorithmicensure}{\textbf{输出:}\unskip}

\begin{algorithm}
  \caption{Calculate $y = x^n$}
  \label{alg1}
  \small
  \begin{algorithmic}
    \REQUIRE $n \geq 0$
    \ENSURE $y = x^n$

    \STATE $y \leftarrow 1$
    \STATE $X \leftarrow x$
    \STATE $N \leftarrow n$

    \WHILE{$N \neq 0$}
      \IF{$N$ is even}
        \STATE $X \leftarrow X \times X$
        \STATE $N \leftarrow N / 2$
      \ELSE[$N$ is odd]
        \STATE $y \leftarrow y \times X$
        \STATE $N \leftarrow N - 1$
      \ENDIF
    \ENDWHILE
  \end{algorithmic}
\end{algorithm}

\input{data/chap03}
\input{data/chap04}


% 其他部分
\backmatter

% 参考文献
\bibliography{ref/refs}  % 参考文献使用 BibTeX 编译
% \printbibliography       % 参考文献使用 BibLaTeX 编译

% 附录
% 本科生需要将附录放到声明之后,个人简历之前
\appendix
% \input{data/appendix-survey}       % 本科生:外文资料的调研阅读报告
% \input{data/appendix-translation}  % 本科生:外文资料的书面翻译
% !TeX root = ../thuthesis-example.tex

\chapter{补充内容}

附录是与论文内容密切相关、但编入正文又影响整篇论文编排的条理和逻辑性的资料,例如某些重要的数据表格、计算程序、统计表等,是论文主体的补充内容,可根据需要设置。

附录中的图、表、数学表达式、参考文献等另行编序号,与正文分开,一律用阿拉伯数字编码,
但在数码前冠以附录的序号,例如“图~\ref{fig:appendix-figure}”,
“表~\ref{tab:appendix-table}”,“式\eqref{eq:appendix-equation}”等。


\section{插图}

% 附录中的插图示例(图~\ref{fig:appendix-figure})。

\begin{figure}
  \centering
  \includegraphics[width=0.6\linewidth]{example-image-a.pdf}
  \caption{附录中的图片示例}
  \label{fig:appendix-figure}
\end{figure}


\section{表格}

% 附录中的表格示例(表~\ref{tab:appendix-table})。

\begin{table}
  \centering
  \caption{附录中的表格示例}
  \begin{tabular}{ll}
    \toprule
    文件名          & 描述                         \\
    \midrule
    thuthesis.dtx   & 模板的源文件,包括文档和注释 \\
    thuthesis.cls   & 模板文件                     \\
    thuthesis-*.bst & BibTeX 参考文献表样式文件    \\
    thuthesis-*.bbx & BibLaTeX 参考文献表样式文件  \\
    thuthesis-*.cbx & BibLaTeX 引用样式文件        \\
    \bottomrule
  \end{tabular}
  \label{tab:appendix-table}
\end{table}


\section{数学表达式}

% 附录中的数学表达式示例(式\eqref{eq:appendix-equation})。
\begin{equation}
  \frac{1}{2 \uppi \symup{i}} \int_\gamma f = \sum_{k=1}^m n(\gamma; a_k) \mathscr{R}(f; a_k)
  \label{eq:appendix-equation}
\end{equation}


\section{文献引用}

附录中的参考文献引用

\printbibliography


% 致谢
% !TeX root = ../thuthesis-example.tex

\begin{acknowledgements}
  我谨向我的导师刘正伟表示衷心的感谢,他引领我进入了一个全新的数学研究世界。没有他持续的支持、鼓励、耐心和专业建议,这项工作无法完成。

  我还要要感谢数学中心的魏朝晖、刘子文、刘锦鹏、丁大威老师他们在同我的讨论中给予了我很多帮助。
  以及雁栖湖应用中心的程嵩老师,这项工作的一部分是在他的指导下完成的。


  最后,我要感谢我的伙伴们——陆凡、魏付川、张浩、赵子硕、李俊峰、阮钰泽、王宁烽、卢润迪、何会萱、张睿齐,以及清华的朋友们,感谢他们与我一起讨论了许多精彩的数学问题。
\end{acknowledgements}


% 声明
% 本科生开题报告不需要
\statement
% 将签字扫描后的声明文件 scan-statement.pdf 替换原始页面
% \statement[file=scan-statement.pdf]
% 本科生编译生成的声明页默认不加页脚,插入扫描版时再补上;
% 研究生编译生成时有页眉页脚,插入扫描版时不再重复。
% 也可以手动控制是否加页眉页脚
% \statement[page-style=empty]
% \statement[file=scan-statement.pdf, page-style=plain]

% 个人简历、在学期间完成的相关学术成果
% 本科生可以附个人简历,也可以不附个人简历
% !TeX root = ../thuthesis-example.tex

\begin{resume}

  \section*{个人简历}

  1997 年 11 月 11 日出生于四川省自贡市。

  2016 年 9 月考入华中科技大学数学与统计学院数学与应用数学专业,2020 年 6 月本科毕业并获得理学学士学位。

  2020 年 9 月免试进入清华大学数学科学系攻读数学博士至今。


  \section*{在学期间完成的相关学术成果}

  \subsection*{学术论文}

  \begin{achievements}
    \item Shao, Y., Wei, F., Cheng, S., \& Liu, Z. (2024). Simulating noisy variational quantum algorithms: A polynomial approach. Physical Review Letters, 133(12), 120603.
    \item Sun, W., Wei, F., Shao, Y., \& Wei, Z. (2024). Sudden death of quantum advantage in correlation generations. Science Advances, 10(47), eadr5002.
    \item Huang, Y., Shao, Y., Ren, W., Sun, J., \& Lv, D. (2023). Efficient quantum imaginary time evolution by drifting real-time evolution: An approach with low gate and measurement complexity. Journal of Chemical Theory and Computation, 19(13), 3868-3876.
    \item Shao, Y., Li, Y., Wei, F., Zhan, H., Wang, B., Wei, Z., Zhang, L., \& Liu, Z. (2024). Variational Graphical Quantum Error Correction Codes: adjustable codes from topological insights. arXiv preprint arXiv:2410.02608.
    \item Zhang, R., Shao, Y., Wei, F., Cheng, S., Wei, Z., \& Liu, Z. (2024). Clifford Perturbation Approximation for Quantum Error Mitigation. arXiv preprint arXiv:2412.09518.
  \end{achievements}

  %\subsection*{专利}

  %\begin{achievements}
    %\item 任天令, 杨轶, 朱一平, 等. 硅基铁电微声学传感器畴极化区域控制和电极连接的方法: 中国, CN1602118A[P]. 2005-03-30.
    %\item Ren T L, Yang Y, Zhu Y P, et al. Piezoelectric micro acoustic sensor based on ferroelectric materials: USA, No.11/215, 102[P]. (美国发明专利申请号.)
  %\end{achievements}

\end{resume}


% 指导教师/指导小组评语
% 本科生不需要
% !TeX root = ../thuthesis-example.tex

\begin{comments}
% \begin{comments}[name = {指导小组评语}]
% \begin{comments}[name = {Comments from Thesis Supervisor}]
% \begin{comments}[name = {Comments from Thesis Supervision Committee}]

  %论文提出了……
  暂无

\end{comments}


% 答辩委员会决议书
% 本科生不需要
% !TeX root = ../thuthesis-example.tex

\begin{resolution}
  经典模拟是研究量子算法、量子纠错、校准量子计算机的重要工具。现如今随着量子硬件的发展,传统的经典模拟方法已经逐渐逼近性能极限。在当前的时代背景下,变分量子算法因为不需要昂贵的纠错协议受到了广泛关注,但变分量子算法能否在线路噪声等实际环境下实现量子优势仍然是个困难的、被反复讨论的开放问题。

邵钰菓的博士学位论文《基于Pauli路径积分模拟量子线路》聚焦NISQ时代量子线路的经典模拟难题,提出了基于Pauli路径积分的泡利基下反向路径积分(OBPPP)算法并拓展了稀疏泡利动力学(SPD)算法,系统分析了算法的计算复杂度与误差特性,揭示了噪声对量子优势边界的临界影响。论文的主要成果如下:


1. 证明了绝大多数含噪声变分量子线路观测期望值在一定条件下的经典可模拟性。

2. 证明了在Pauli噪声率恒定或高于1/logL时,观测期望值可被多项式复杂度经典模拟;而当噪声率低于1/L时,模拟复杂度可能指数增长。

3. 通过复现IBM 127量子比特Eagle处理器的Ising模型实验,验证了该模拟算法在精度与效率上显著优于实际量子设备,并实现了噪声强度的量化评估。

研究兼具理论深度与实践价值,为量子硬件验证及噪声抑制提供了重要工具。


论文选题前沿,逻辑严谨,成果发表于《Physical Review Letters》等高水平期刊,创新性突出,理论意义与应用价值显著。答辩过程中,邵钰菓陈述清晰,表达严谨,准确回答了委员会提出的问题,展现了扎实的专业功底与独立科研能力。


经答辩委员会讨论与无记名投票,一致通过邵钰菓的博士学位论文答辩,认为这是一篇优秀的博士学位论文,建议授予邵钰菓理学博士学位。


\end{resolution}


% 本科生的综合论文训练记录表(扫描版)
% \record{file=scan-record.pdf}

\end{document}
