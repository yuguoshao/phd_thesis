\chapter{具体线路的复杂度与误差分析}

\section{变分量子线路}
在如今NISQ时代,噪声难以避免的出现在量子线路中。这些噪声导致许多需要高保真度的量子算法难以实现。比如Shor算法需要使用到深层量子线路,因此对噪声的容忍度较低。为了在近期的量子设备中实现实用性的量子算法,变分量子算法作为最受欢迎的NISQ算法,普遍被认为是一个很好的解决方案。变分量子算法的核心思想是通过参数化的量子线路作为
%#TODO

在本节的讨论中,我们将变分量子线路的深度记为 $L$,即 $\mathcal{U}(\bm{\theta})=\mathcal{U}_L(\bm{\theta}_L)\cdots\mathcal{U}_1(\bm{\theta}_1)$。变分量子线路的每一层 $\mathcal{U}_i(\bm{\theta}_i)$ 是由 $R_i$ 个旋转门和 $C_i$ 个 Clifford 门组成,这些门作用于相互不相交的量子比特上,$\bm{\theta}_i=(\theta_{i,1},\cdots,\theta_{i,R_i})$ 表示第 $i$ 层的参数向量,$\bm{\theta}=(\bm{\theta}_1,\cdots,\bm{\theta}_L)$ 表示整个变分量子线路的参数向量。
具体来说,第 $i$ 层中的第 $j$ 个旋转门表示为 $U_{i, j}(\theta_{i,j})=\exp{-i \frac{\theta_{i,j}}{2} \sigma_{i,j}}$,其中 $j \in \{1, \cdots , R_i\}$,$\theta_{i,j}$ 是旋转角度,$\sigma_{i, j}\in \{\mathbb{I}, X,Y,Z\}^{\otimes n}$为Pauli旋转门对应的Pauli算符。
类似地,第 $i$ 层中的第 $k$ 个 Clifford 门表示为 $V_{i,k}$,其中 $k \in \{1, \cdots , C_i\}$。$V_{i,k}\in\{\mathrm{H}(a),\mathrm{S}(a),\mathrm{CNOT}(a,b)\}$,其中 $a,b $ 指的是门作用的量子比特的序号。

记所有Pauli旋转门 $U_{i,j}(\theta_{i,j})$ 的 Pauli 算符集合为 $\{\sigma_{i,j}\}$。
我们将集合 $\{\overline{\sigma}_{i,j}\}$ 表示为经过后续 Clifford 门变换后的所有 $\sigma_{i,j}$,即 $\overline{\sigma}_{i,j}= \mathcal{V}_{L} \cdots \mathcal{V}_{i} \sigma_{i,j} \mathcal{V}_{i}^\dagger \cdots \mathcal{V}_{L}^\dagger$,其中 $\mathcal{V}_{i} = \prod_{k=1}^{C_i} V_{i,k}$ 是对应于第 $i$ 层中所有 Clifford 门的张量积的幺正变换。

为了确保引理~\ref{lemma:cross_items} 的有效性,我们需要一个容易实现的前提条件:假设集合 $\{\overline{\sigma}_{i,j}\}$ 在忽略相位的意义下,可以生成 $\{\mathbb{I}, X, Y, Z\}^{\otimes n}$。这一条件可以通过以下等式来表示:
\begin{equation}\label{eq:generate}
  \langle \{\overline{\sigma}_{i,j}\}\rangle/\left(\langle \{\overline{\sigma}_{i,j}\}\rangle\cap\langle i\mathbb{I}^{\otimes n}\rangle\right)=\{\mathbb{I},X,Y,Z\}^{\otimes n},
\end{equation}
这里 $\langle \{\overline{\sigma}_{i,j}\} \rangle$ 指的是由集合 $\{\overline{\sigma}_{i,j}\}$ 生成的 Pauli 子群,这意味着 $\langle \{\overline{\sigma}_{i,j}\}\rangle$ 中的元素可以表示为 $\{\overline{\sigma}_{i,j}\}$ 中元素的有限乘积。
为了证明这一条件确实容易满足,考虑在电路的最后两层中,每个量子比特上都有一层 $R_X$ 门和一层 $R_Z$ 门作用,那么 $\{X_i, Z_i\}_{i=1,\cdots,n}$ 包含在 $\{\overline{\sigma}_{i,j}\}$ 中。这些 $\{ X_i,Z_i \}_{i=1,\cdots,n}$ 足以生成 $\{\mathbb{I},X,Y,Z\}^{\otimes n}$。事实上,这一充分条件可以进一步减弱,如第~\ref{temp:subsubsection:cross_items}节 中所示。


\subsection{计算复杂度分析}
\begin{proposition}\label{prop:layer_terms}
    计算 $f$ 中第 $i$ 层项的时间和空间复杂度为 $\order{n}$,可通过如下等式计算:
    \begin{equation}\label{eq:i-layer_terms}
    \begin{aligned}
    \Tr{s_i\mathcal{U}_i s_{i-1}\mathcal{U}_i^\dagger}&=\Tr{\left(s_i s_{i-1}\right)\big|_{I_i}}\\
    \prod_{k=1}^{C_i}&\Tr{\left(s_iV_{i,k} s_{i-1}V_{i,k}^{\dagger}\right)\big|_\g{V_{i,k}}}\prod_{\sigma_{i,j}\in C(i,s_{i-1})}\Tr{\left(s_i s_{i-1}\right)\big|_\g{\sigma_{i,j}}}\\
    \prod_{\sigma_{i,j'}\in AC(i,s_{i-1})} &\Bigg\{ \Tr{\left(s_i s_{i-1}\right)\big|_\g{\sigma_{i,j'}}}\cos{\theta_{i,j'}}- \Tr{\left(is_i \sigma_{i,j'}s_{i-1}\right)\big|_\g{\sigma_{i,j'}}}\sin{\theta_{i,j'}}  \Bigg\},
    \end{aligned}
    \end{equation}
    其中$\supp:\{\mathbb{I}, X, Y, Z\}^{\otimes n}\cup\{\mathrm{CNOT}_{a,b},\mathrm{H}_a,\mathrm{S}_a\}\rightarrow2^{\{1,\cdots, n\}}$ 为从幺正算符到其非平凡作用的量子比特的映射。这里 $2^{\{1,\cdots, n\}}$ 表示 $\{1,\cdots, n\}$ 的所有子集。
    为了简化,我们将 $i$ 层中 $n$ 个量子比特的索引根据所应用门的类型分为三组。这些组分别表示为符号 $\big|_{I_i}$、$\big|_\g{V_{i,k}}$ 和 $\big|_\g{\sigma_{i,j}}$,对应于没有门作用、Clifford 门作用和 Pauli 旋转门作用。此外,集合 $C(i,s_{i-1})$ 和 $AC(i,s_{i-1})$ 分别表示 $\{\sigma_{i,j}\}_{j=1}^{R_i}$ 中与 $s_{i-1}$ 对易和反对易的 Pauli 算符集合。
    
    \begin{proof}
    
    在变分量子线路中,$\mathcal{U}_i$ 由一系列门 $U_{i,1},\cdots,U_{i,R_i}$ 和 $V_{i,1},\cdots,V_{i,C_i}$ 组成,并且不会在每个量子比特上操作两次。
    Pauli 旋转门 $U_{i,j}(\theta_{i,j})$ 的形式为:
    \begin{equation}
      U_{i,j}(\theta_{i,j})=\exp{-i \frac{\theta_{i,j}}{2} \sigma_{i,j}}.
    \end{equation}
    
    Clifford 门 $V_{i,k}$ 可以从 $\{\mathrm{H}(a),\mathrm{S}(a),\mathrm{CNOT}(a,b) \}_ {a\neq b}$ 中选择。
    因此,我们有:
    \begin{equation}
      \begin{aligned}
        \Tr{s_i\mathcal{U}_i s_{i-1}\mathcal{U}_i^\dagger}=&\Tr{\left(s_i s_{i-1}\right)\big|_{I_i}}
        \prod_{k=1}^{C_i}\Tr{\left(s_iV_{i,k} s_{i-1}V_{i,k}^{\dagger}\right)\big|_\g{V_{i,k}}}\\
        &\prod_{j=1}^{R_i}\Tr{\left(s_i U_{i,j}(\theta_{i,j}) s_{i-1} U_{i,j}^\dagger(\theta_{i,j})\right) \big|_\g{\sigma_{i,j}}}.
      \end{aligned}
    \end{equation}
    
    厄米 算符 $X$ 的指数定义为 $\exp{X}=\sum_{k=0}^\infty\frac{X^k}{k!}$。对于 Pauli 算符,任何 Pauli 算符号的满足自反性 $\sigma^2=\mathbb{I}$,因此我们有:
    \begin{equation}
      \exp{-i \frac{\theta}{2} \sigma}=\sum_{k=0}^\infty\frac{(-i \frac{\theta}{2}\sigma)^k}{k!}=\sum_{k=0}^\infty\frac{(-1)^k (\frac{\theta}{2})^{2k}}{(2k)!} \mathbb{I}- i\frac{(-1)^k (\frac{\theta}{2})^{2k+1}}{(2k+1)!}\sigma=\cos{\frac{\theta}{2}}\mathbb{I}-i \sin{\frac{\theta}{2}}\sigma.
    \end{equation}
    
    根据 Pauli算符$\sigma$ 和另一个 Pauli算符 $\sigma'$ 的对易/反对易关系,有:
    \begin{equation}
    \begin{aligned}
    &\sigma'\exp{-i \frac{\theta}{2} \sigma}=\exp{-i \frac{\theta}{2} \sigma}\sigma',\text{ 如果 }\sigma\text{ 与 }\sigma'\text{ 对易。}\\
    &\sigma'\exp{-i \frac{\theta}{2} \sigma}=\exp{i \frac{\theta}{2} \sigma}\sigma',\text{ 如果 }\sigma\text{ 与 }\sigma'\text{ 反对易。}\\
    \end{aligned}
    \end{equation}
    因此可以将 $\{\sigma_{i,j}\}$ 分为两种情况。如果 $\sigma_{i,j}$ 与 $s_{i-1}$ 对易,有:
    
    \begin{equation}
      \begin{aligned}
      \Tr{\left(s_i U_{i,j}(\theta_{i,j}) s_{i-1} U_{i,j}^\dagger(\theta_{i,j})\right)\big|_\g{\sigma_{i,j}}}
      &=\Tr{(s_i \exp{-i \frac{\theta_{i,j}}{2}\sigma_{i,j} }  \exp{i \frac{\theta_{i,j}}{2}\sigma_{i,j} }s_{i-1}) \big|_\g{\sigma_{i,j}}}\\
      &=\Tr{(s_i \exp{-i \frac{\theta_{i,j}}{2}\sigma_{i,j} } \exp{i \frac{\theta_{i,j}}{2}\sigma_{i,j} }s_{i-1}) \big|_{\g{\sigma_{i,j}}}}\\
      &=\Tr{(s_i s_{i-1})\big|_{\g{\sigma_{i,j}}}}.
    \end{aligned}
    \end{equation}
    而若 $\sigma_{i,j}$ 与 $s_{i-1}$ 反对易,则有:
    \begin{equation}
      \begin{aligned}
      &\Tr{\left(s_i U_{i,j}(\theta_{i,j}) s_{i-1} U_{i,j}^\dagger(\theta_{i,j})\right)\big|_\g{\sigma_{i,j}}}\\
      &=\Tr{ \left(s_i\exp{-i \frac{\theta_{i,j}}{2}\sigma_{i,j} } s_{i-1}\exp{i \frac{\theta_{i,j}}{2}\sigma_{i,j}}\right) \big|_\g{\sigma_{i,j}}}\\
      &=\Tr{\left( s_i  \exp{-i\theta_{i,j}\sigma_{i,j}} s_{i-1}\right) \big|_\g{\sigma_{i,j}}}\\
      &=\Tr{ \left(s_i (\cos{\theta_{i,j}}\mathbb{I}-i\sin{\theta_{i,j}}\sigma_{i,j}) s_{i-1} \right) \big|_\g{\sigma_{i,j}}}\\
      &=\Tr{\left(s_i s_{i-1}\right)\big|_\g{\sigma_{i,j}}}\cos{\theta_{i,j}}-\Tr{\left(i s_i \sigma_{i,j}s_{i-1}\right)\big|_\g{\sigma_{i,j}}}\sin{\theta_{i,j}}.
    \end{aligned}
    \end{equation}
    以上完成了 式.~\eqref{eq:i-layer_terms} 的证明。

    此外,对于计算复杂度,可以分类讨论。
    对$\big|_{I_i}$项,需要比较$s_i$ 和 $s_{i-1}$ 在$\supp(I_i)$上是否相同。这一操作的时间和空间复杂度为 $\order{n}$。
    对于$\big|_\g{V_{i,k}}$项,需要比较$s_i$ 和 $V_{i,k}s_{i-1}V_{i,k}^\dagger$ 在$\supp(V_{i,k})$上是否相同,因为 $V_{i,k}\in\{\mathrm{H}(a),\mathrm{S}(a),\mathrm{CNOT}(a,b)\}$,$V_{i,k}s_{i-1}V_{i,k}^\dagger$ 可以在 $\order{1}$ 的时间内计算。因此,对于所有$\big|_\g{V_{i,k}}$项,总的时间和空间复杂度为 $\order{n}$。
    对于$\big|_\g{\sigma_{i,j}}$项,需要首先比较$s_{i-1}$ 和 $\sigma_{i,j}$ 在$\supp(\sigma_{i,j})$上的对易关系,然后再比较$s_i$ 和 $s_{i-1}$(或是$\sigma_{i,j'}s_{i-1}$) 在$\supp(\sigma_{i,j})$上是否相同。这两个操作的时间和空间复杂度为均为 $\order{\abs{\supp(\sigma_{i,j})}}$。因此,对于所有$\big|_\g{\sigma_{i,j}}$项,总的时间和空间复杂度为 $\order{n}$。

    综上所述,计算 $f$ 中第 $i$ 层项的时间和空间复杂度为 $\order{n}$。
    \end{proof}
\end{proposition}

\begin{remark}\label{remark:f_ele}
通过利用不同 Pauli 算符的正交性,可以在 $\Tr{s_i\mathcal{U}_i s_{i-1}\mathcal{U}_i^\dagger} \neq 0$ 时,建立以下$s_i$ 和 $s_{i-1}$ 之间的关系:

\begin{tabular}{|c|c|c|}
  \hline
   项 & 关系 & $f$ 中的因子\\
  \hline
  ${I_i}$ &$s_{i-1}\big|_{I_i} = s_{i}\big|_{I_i}$& 1 \\
  \hline
  $V_{i,k}$ &$s_{i-1}\big|_\g{V_{i,k}}= \pm V_{i,k}^{\dagger} s_{i}V_{i,k}\big|_\g{V_{i,k}}$& $\pm 1$ \\
  \hline
  $C(i,s_{i-1})=C(i,s_{i})$&$s_{i}|_\g{\sigma_{i,j}}=s_{i-1}|_\g{\sigma_{i,j}}$& 1 \\
  \hline
  \multirow{2}{*}{$AC(i,s_{i-1})=AC(i,s_{i})$}
  &$s_{i}|_\g{\sigma_{i,j}}=s_{i-1}|_\g{\sigma_{i,j}}$& $\cos{\theta_{i,j}}$ \\
  \cline{2-3}
  &$s_{i}|_\g{\sigma_{i,j}}=\pm i \sigma_{i,j} s_{i-1}|_\g{\sigma_{i,j}}$& $\mp \sin{\theta_{i,j}}$ \\
  \hline
\end{tabular}

以上表格中的关系可以帮助我们计算 $f$ 中的每一项。以$I_i$ 为例,该表格表面若$\bm{s}$有非平凡贡献$f(\mathcal{U},\bm{s},O,\rho) \neq 0$,则必有$s_{i-1}\big|_{I_i} = s_{i}\big|_{I_i}$。这意味着 $s_{i-1}$ 和 $s_{i}$ 在第 $i$ 层的第 $j$ 个旋转门上的作用是相同的。且这一关系为$f$带来了一个因子1。

通过命题~\ref{prop:layer_terms},可以得知对于变分量子线路中的每一层,计算 $f$ 中的每一项的时间和空间复杂度为 $\order{n}$。因此,计算 $f$ 中的所有项的时间复杂度为 $\order{L n}$,得到式~\eqref{eq:cost_f}中$C_{f\mathcal{U}}=\order{L n}$。式~\eqref{eq:space_f}中的$S_{f\mathcal{U}}=\order{n}$。
\end{remark}

结合式~\eqref{eq:cost_rho}和式~\eqref{eq:cost_O},我们可以得到式~\eqref{eq:cost_f}对于变分量子线路取值为:
\begin{equation}
    C_f=\order{L n}+\order{n}+\order{\poly(n)}=\order{\poly(n)+L n}.
\end{equation}
式~\eqref{eq:space_f}对于变分量子线路取值为:
\begin{equation}
    S_f=\order{n}+\order{\poly(n)}+\order{\poly(n)}=\order{\poly(n)}.
\end{equation}


在OBPPP算法的复杂度的分析中,还涉及到对一个给定的Pauli算符$s$和构成线路$\mathcal{U}$的门集合$\{U_{i,j}\}$,求解以下集合:
\begin{equation}
    \left\{ s'\mid \Tr{s' U^\dagger_{i,j}s U_{i,j}}\neq 0 \right\}.
\end{equation}
为了估计整个算法的时间复杂度,我们需要考虑这一集合的最大元素数$N_{\text{max}}$、求解该集合所需要的时间$C_{\text{split}}$和空间复杂度$S_{\text{split}}$。

对于在变分线路中,每个门可最大分裂的Pauli 算符数量$N_{\max}$:
\begin{equation}
    N_{\text{max}} = \max_{i,j,s}\abs{ \left\{ s'\mid \Tr{s' U^\dagger_{i,j}s U_{i,j}}\neq 0 \right\}}.
\end{equation}
注意到,当 $U_{i,j}$ 为Clifford 门时,因为Clifford 门作用于Pauli 算符后仍为Pauli 算符,因此Clifford门不会增加Pauli 算符的数量。而当 $U_{i,j}$ 为Pauli 旋转门时,假设旋转的Pauli算符为$\sigma$。根据注释~\ref{remark:f_ele}当$s$与$\sigma$对易时,$s'=s$,因此最大分裂个数为1。当$s$与$\sigma$反对易时,$s'=s$ 或 $s'=\pm i \sigma s$,因此最大分裂个数为2。因此,$N_{\text{max}}=2$。

对于求解该集合所需要的时间$C_{\text{split}}$和空间复杂度$S_{\text{split}}$,对于Clifford门可以以$\order{1}$的时间和空间复杂度求解。对于Pauli 旋转门,需要比较$s$和$s'$的对易关系,并输出$s'$。这一操作的时间和空间复杂度为 $\order{n}$。因此,$C_{\text{split}}=\order{n}$,$S_{\text{split}}=\order{n}$。

因此,对于变分量子线路,对于\textbf{情况1}根据式~\eqref{eq:case1:N_M}和式~\eqref{eq:case1:cost_s},Pauli路径枚举过程的时间复杂度为:
\begin{equation}
    C_{s}=\mathrm{Poly}(n) \order{L} C_{\text{split}} N_{\text{max}}^{M}= \mathrm{Poly}(n) \order{L} 2^M.
\end{equation}
满足$\abs{\bm{s}}\leq M$的Pauli路径数量最多为:
\begin{equation}
    N_M=\poly(n) N_{\text{max}}^{M}=\poly(n) 2^M.
\end{equation}

对于\textbf{情况2},根据式~\eqref{eq:case2:N_M}和式~\eqref{eq:case2:cost_s},Pauli路径枚举过程的时间复杂度为:
\begin{equation}
    C_{s}=\mathrm{Poly}(n) \order{(nL)^M} C_{\text{split}} N_{\text{max}}^{M}= \mathrm{Poly}(n) \order{(nL)^M} 2^M.
\end{equation}
满足$\abs{\bm{s}}\leq M$的Pauli路径数量最多为:
\begin{equation}
    N_M=\poly(n)(nLN_{\text{max}})^{M+1}= \poly(n)(2nL)^{M+1}.
\end{equation}

对于枚举过程的空间复杂度,根据式~\eqref{eq:path:space_cost}:
\begin{equation}
    S_{s}=\order{\mathrm{Poly}(n)+\max\{S_{\text{split}},n\}L}=\order{\mathrm{Poly}(n)+nL}.
\end{equation}


综上所述,对于变分量子线路,根据式~\eqref{eq:total_cost}和式~\eqref{eq:space_cost},在给定截断参数$M$的情况下,含噪声期望值模拟的时间复杂度$C$和空间复杂度$S$分别为:
\begin{equation}\label{eq:vqa:cost}
    \begin{aligned}
        C&=\begin{cases}
            \mathrm{Poly}(n) \order{L} 2^M, & \text{如果$\abs{\bullet}_{\mathcal{N}}$属于\textbf{情况1}} \\
           \mathrm{Poly}(n) \order{(2nL)^{M+1}}, & \text{如果$\abs{\bullet}_{\mathcal{N}}$属于\textbf{情况2}}
        \end{cases}\\
        S&=\mathrm{Poly}(n)+nL,
    \end{aligned}
\end{equation}
其中$n$为量子比特数,$L$为线路深度,$\mathcal{N}$为噪声模型。


\subsection{误差分析}
在上一节中,已经得到了在给定截断参数$M$的情况下,模拟变分量子线路的计算复杂度(式~\eqref{eq:vqa:cost})。在这一节中,我们将讨论在给定的误差容忍度$\epsilon$下,如何选择截断参数$M$。

误差可以用均方误差(Mean Square Error, MSE)来衡量,对于一个随机变量$X$,其均方误差定义为$\mathbb{E}[(X-\hat{X})^2]$,其中$\hat{X}$是$X$的估计值。在这里,我们考虑的是含噪声期望值模拟的误差,即真实含噪声期望值$\widetilde{\langle O\rangle}$和其估计值$\widehat{\langle O\rangle}$之间的均方误差。根据定义(式~\eqref{eq:noise_expectation}和式~\eqref{eq:truncated_expectation}),含噪声期望值模拟的误差为:
\begin{equation}
    \widetilde{\langle O\rangle}-\widehat{\langle O\rangle}=\sum_{\abs{\bm{s}}_{\mathcal{N}}>M}\tilde{f}(\mathcal{U},\bm{s},O,\rho),
\end{equation}
其中$\abs{\bullet}_{\mathcal{N}}$定义在式~\eqref{eq:noise_hamming_weight}中,表示与噪声模型$\mathcal{N}$关联的Hamming weight,$\tilde{f}(\mathcal{U},\bm{s},O,\rho)$定义在式~\eqref{eq:noise_contribution}中,表示一个Pauli路径$\bm{s}$对含噪声期望值的贡献。


如果有式~\eqref{eq:generate}成立,有以下引理成立:
\begin{lemma}\label{lemma:cross_items}
    对于变分量子线路,如果式~\eqref{eq:generate}成立,则对任意两个不同的Pauli路径$\bm{s}$和$\bm{s}'$,有:
    \begin{equation}\label{eq:E_cross_equals_0}
        \mathbb{E}_{\bm{\theta}}\left[\tilde{f}(\mathcal{U},\bm{s},O,\rho)\tilde{f}(\mathcal{U},\bm{s}',O,\rho)\right]=0,
    \end{equation}
    其中期望是对于均匀随机选择的旋转角度$\bm{\theta}$。
\end{lemma}
我们先暂时跳过该引理的证明,我们将在后文中证明该引理。

关于均方误差,有以下引理:
\begin{lemma}\label{lemma:MSE_l}
    对于变分量子线路,如果式~\eqref{eq:E_cross_equals_0}成立,那么对任意的$\nu > 0$,只要截断参数$M$满足:
    \begin{equation}
        M\geq\frac{1}{4\gamma}\ln{\frac{\norm{O}_\infty^2}{\nu}},
    \end{equation}
    其中$\gamma:=\min\{p|{p \in \{p_x,p_y,p_z\},p\neq 0}\}$,那么含噪声期望值模拟的均方误差满足:
    \begin{equation}
        \mathbb{E}_{\bm{\theta}}\left[\left(\widetilde{\langle O\rangle}-\widehat{\langle O\rangle}\right)^2\right]\leq\nu,
    \end{equation}
    其中期望是对于均匀随机选择的旋转角度$\bm{\theta}$。
\end{lemma}
\begin{proof}
    根据式~\eqref{eq:noise_contribution},含噪声期望值模拟的误差为:
    \begin{equation}
        \widetilde{\langle O\rangle}-\widehat{\langle O\rangle}=\sum_{\abs{\bm{s}}_{\mathcal{N}}>M}\tilde{f}(\mathcal{U},\bm{s},O,\rho).
    \end{equation}
    因此,含噪声期望值模拟的均方误差为:
    \begin{equation}
        \begin{aligned}
            \mathbb{E}_{\bm{\theta}}\left[\left(\widetilde{\langle O\rangle}-\widehat{\langle O\rangle}\right)^2\right]&=\mathbb{E}_{\bm{\theta}}\left[\left(\sum_{\abs{\bm{s}}_{\mathcal{N}}>M}\tilde{f}(\mathcal{U},\bm{s},O,\rho)\right)^2\right]\\
            &=\sum_{\abs{\bm{s}}_{\mathcal{N}}>M}\mathbb{E}_{\bm{\theta}}\left[\tilde{f}(\mathcal{U},\bm{s},O,\rho)^2\right],
        \end{aligned}
    \end{equation}
    其中,等号是因为根据式~\eqref{eq:E_cross_equals_0},对于任意两个不同的Pauli路径$\bm{s}$和$\bm{s}'$,有:
    \begin{equation}
        \mathbb{E}_{\bm{\theta}}\left[\tilde{f}(\mathcal{U},\bm{s},O,\rho)\tilde{f}(\mathcal{U},\bm{s}',O,\rho)\right]=0.
    \end{equation}
    
    另一方面,在无噪声的理想情况下,有:
    \begin{equation}
        \abs{\sum_{\bm{s}}f(\mathcal{U},\bm{s},O,\rho)}=\abs{\Tr{\rho \mathcal{U}^\dagger s \mathcal{U} O}}\leq\norm{O}_\infty.
    \end{equation}
    因此,有如下不等式:
    \begin{equation}
        \mathbb{E}_{\bm{\theta}}\left[ \sum_{\bm{s}}\tilde{f}(\mathcal{U},\bm{s},O,\rho)\right]^2=\sum_{\bm{s}}\mathbb{E}_{\bm{\theta}}\left[f(\mathcal{U},\bm{s},O,\rho)^2\right]\leq\norm{O}_\infty^2.
    \end{equation}
    根据不等式~\eqref{eq:noise_suppression},可以得到:
    \begin{equation}\label{temp:eq:noise_suppression}
        \begin{aligned}
            \sum_{\abs{\bm{s}}_{\mathcal{N}}>M}\mathbb{E}_{\bm{\theta}}\left[\tilde{f}(\mathcal{U},\bm{s},O,\rho)^2\right] &\leq (1-2\gamma)^{2M} \sum_{\abs{\bm{s}}_{\mathcal{N}}>M}\mathbb{E}_{\bm{\theta}}\left[f(\mathcal{U},\bm{s},O,\rho)^2\right]\\
            &\leq (1-2\gamma)^{2M} \norm{O}_\infty^2\\
            &\leq \exp{-4\gamma M} \norm{O}_\infty^2.
        \end{aligned}
    \end{equation}

    于是,对于任意的$\nu > 0$,只要截断参数$M$满足:
    \begin{equation}
        M\geq\frac{1}{4\gamma}\ln{\frac{\norm{O}_\infty^2}{\nu}},
    \end{equation}
    就有:
    \begin{equation}
        \mathbb{E}_{\bm{\theta}}\left[\left(\widetilde{\langle O\rangle}-\widehat{\langle O\rangle}\right)^2\right]\leq\nu.
    \end{equation}
\end{proof}
\begin{remark}
    特别的,对于退极化噪声模型,有$p_x=p_y=p_z=\frac{\lambda}{4}$,因此根据引理~\ref{lemma:noise},可以得到:
    \begin{equation}
        \tilde{f}(\mathcal{U},\bm{s},O,\rho)=(1-\lambda)^{\abs{\bm{s}}}f(\mathcal{U},\bm{s},O,\rho).
    \end{equation}
    类似于式~\eqref{temp:eq:noise_suppression},可以得到:
    \begin{equation}
        \mathbb{E}_{\bm{\theta}}\left[\left(\widetilde{\langle O\rangle}-\widehat{\langle O\rangle}\right)^2\right] \leq (1-\lambda)^{2M} \norm{O}_\infty^2 \leq \exp{-2\lambda M} \norm{O}_\infty^2.
    \end{equation}
    因此,对于退极化噪声模型,只要截断参数$M$满足:
    \begin{equation}
        M\geq\frac{1}{2\lambda}\ln{\frac{\norm{O}_\infty^2}{\nu}},
    \end{equation}
    就有$\mathbb{E}_{\bm{\theta}}\left[\left(\widetilde{\langle O\rangle}-\widehat{\langle O\rangle}\right)^2\right]\leq\nu$。
\end{remark}

结合以上误差分析,以及过去的计算复杂度分析(式~\eqref{eq:vqa:cost}),我们可以得到如下推论:

\begin{corollary}\label{corollary:noise}
    对于变分量子线路,如果初始态密度矩阵$\rho$和观测量$O$分别满足稀疏性和Pauli稀疏性,且式~\eqref{eq:E_cross_equals_0}成立,那么对于任意的$\nu > 0$,则对于噪声\textbf{情况1},以多项式时间复杂度$\poly(n)\order{L}\left(\frac{\norm{O}_\infty}{\sqrt{\nu}}\right)^{\order{\frac{1}{\gamma}}}=\poly(n,L,\frac{1}{\sqrt{\nu}},\norm{O}_\infty)$;对于噪声\textbf{情况2},以准多项式时间复杂度$\poly(n)\order{(2nL)^{\ln \frac{\norm{O}_\infty}{\sqrt{\nu}}}}^{\order{\frac{1}{\gamma}}}$,能模拟含噪声期望值$\widetilde{\langle O\rangle}$且均方误差满足$\mathbb{E}_{\bm{\theta}}\left[\left(\widetilde{\langle O\rangle}-\widehat{\langle O\rangle}\right)^2\right]\leq\nu$。空间复杂度为$\order{\mathrm{Poly}(n)+nL}$。其中$\gamma:=\min\{p|{p \in \{p_x,p_y,p_z\},p\neq 0}\}$,$n$为量子比特数,$L$为线路深度。
\end{corollary}

结合Markov不等式,可以得到如下主要定理:
\begin{theorem}\label{thm:main}
    对于变分量子线路,如果初始态密度矩阵$\rho$和观测量$O$分别满足稀疏性和Pauli稀疏性,且式~\eqref{eq:E_cross_equals_0}成立,对于固定的$\gamma:=\min\{p|{p \in \{p_x,p_y,p_z\},p\neq 0}\}$,给定任意截断误差$\varepsilon$,存在一个多项式规模的经典算法来近似含噪声期望值$\widetilde{\langle O\rangle}$,在旋转角度$\bm{\theta}$均匀分布时,至少有$1-\delta$的概率满足$\abs{\widetilde{\langle O\rangle}-\widehat{\langle O\rangle}} \leq \varepsilon$。对于噪声\textbf{情况1},其时间复杂度为$\mathrm{Poly}(n) \order{L} \bigg(\frac{\norm{O}_\infty}{\varepsilon \sqrt{\delta}} \bigg)^{\order{\frac{1}{\gamma}}}$,对于\textbf{情况2},其时间复杂度为$\mathrm{Poly}(n)  \order{(2nL)^{\frac{1}{2\gamma} \ln{\frac{\norm{O}_\infty}{\varepsilon \sqrt{\delta}}}+1}}$。空间复杂度为$\order{\mathrm{Poly}(n)+nL}$。
\end{theorem}
\begin{proof}

  根据引理~\ref{lemma:MSE_l},我们已经证明如果 $M\geq\frac{1}{4\gamma}\ln{\frac{\norm{O}_\infty^2}{\nu}}$,则
 \begin{equation}
  \mathbb{E}_{\bm{\theta}}\abs{\widetilde{\langle O\rangle}-\widehat{\langle O\rangle}}^2 \leq \nu.
 \end{equation}
  
 根据Markov不等式,对于任意概率 $\delta$,我们有
 \begin{equation}
  \Pr{\abs{\widetilde{\langle O\rangle}-\widehat{\langle O\rangle}}\geq\frac{1}{\sqrt{\delta}}\sqrt{\mathbb{E}_{\bm{\theta}}\abs{\widetilde{\langle O\rangle}-\widehat{\langle O\rangle}}^2}}=\Pr{\abs{\widetilde{\langle O\rangle}-\widehat{\langle O\rangle}}^2\geq\frac{1}{\delta}{\mathbb{E}_{\bm{\theta}}\abs{\widetilde{\langle O\rangle}-\widehat{\langle O\rangle}}^2}}\leq \delta。
 \end{equation}

因此,对于均匀选取的旋转角度 $\bm{\theta}$,至少有 $1-\delta$ 的概率满足
\begin{equation}
  \abs{\widetilde{\langle O\rangle}-\widehat{\langle O\rangle}} \leq \frac{1}{\sqrt{\delta}}\sqrt{\mathbb{E}_{\bm{\theta}}\abs{\widetilde{\langle O\rangle}-\widehat{\langle O\rangle}}^2}\leq \sqrt{\frac{\nu}{\delta}}.
\end{equation}
设 $\varepsilon$ 为期望误差,可以设置 $\nu=\varepsilon^2 \delta$ 以满足要求,因此截断权重 $M$ 需要满足
\begin{equation}
  M\geq \frac{1}{2\gamma} \ln{\frac{\norm{O}_\infty}{\varepsilon \sqrt{\delta}}}.
\end{equation}

根据式~\ref{eq:vqa:cost},我们可以获得截断权重为 $M=\frac{1}{2\gamma} \ln{\frac{\norm{O}_\infty}{\varepsilon \sqrt{\delta}}}$ 的近似观测值 $\widehat{\langle O\rangle}$。
对于\textbf{情况1},获得 $\widehat{\langle O\rangle}$ 的时间复杂度为:
\begin{equation}
\begin{aligned}
\mathrm{Poly}(n) \order{L} 2^{M}
&=\mathrm{Poly}(n) \order{L} \bigg(\frac{\norm{O}_\infty}{\varepsilon \sqrt{\delta}} \bigg)^{\order{\frac{1}{\gamma}}}\\
&=\mathrm{Poly}\left(n,L,\frac{1}{\varepsilon},\frac{1}{\sqrt{\delta}},\norm{O}_{\infty}\right).
\end{aligned}
\end{equation}

对于\textbf{情况2},所需的时间复杂度约为:
\begin{equation}
  \mathrm{Poly}(n)\order{(2nL)^{M+1}}
  =
  \mathrm{Poly}(n)  \order{(2nL)^{\frac{1}{2\gamma} \ln{\frac{\norm{O}_\infty}{\varepsilon \sqrt{\delta}}}+1}}.
\end{equation}

变量 $n$、$L$、$\varepsilon$、$\delta$ 和 $\norm{O}_\infty$ 之间的关系是非多项式的。因此,在这种情况下,多项式关系不适用,只能得出准多项式关系。
特别地,在给定相对误差 $\frac{\varepsilon}{\norm{O}_\infty}$ 和成功概率 $\delta$ 的情况下,时间复杂度受限于 $\mathrm{Poly}(n)  (2nL)^{\order{\frac{1}{\gamma}}}$,这表明当所需精度保持不变时,计算复杂度与电路规模保持多项式关系。

在空间复杂度方面,根据式~\eqref{eq:vqa:cost},所需的空间复杂度为 $\order{\mathrm{Poly}(n)+nL}$。
\end{proof}

\begin{remark}
    从以上定理可以看出,对于含噪声期望值模拟,无论处于\textbf{情况1}还是\textbf{情况2},其时间复杂度关于线路规模都是多项式级别的$\poly(n,L)$。因此,该算法是一个多项式时间复杂度的经典算法。那么什么时候会体现出量子优势呢?随着实验精度的提高,噪声会进一步被抑制。在这种情况下,可以进一步增加量子线路的深度,以获得更好的结果。因此,一个值得注意的问题是,当噪声率和线路深度的关系如何影响量子优势的体现。
    我们通过以下命题,在噪声处于\textbf{情况1}的情况下,讨论了该问题:
\end{remark}
\begin{proposition}
    \label{prop:lambda_and_L}
    对于变分量子线路,如果初始态密度矩阵$\rho$和观测量$O$分别满足稀疏性和Pauli稀疏性,且式~\eqref{eq:E_cross_equals_0}成立,对于噪声\textbf{情况1},假设$\norm{O}_\infty$是固定的。
    为了估计$\widetilde{\langle O\rangle}(\bm{\theta})$使得$\mathbb{E}_{\bm{\theta}}\abs{\widetilde{\langle O\rangle}-\widehat{\langle O\rangle}}^2$小于一个足够小的常数,有:
    \begin{enumerate}
        \item 如果$\gamma=\Omega(\frac{1}{\log{L}})$,存在一个经典算法可以在$\mathrm{Poly}\left(n,L\right)$时间内完成计算。
        \item 如果$\gamma=\order{\frac{1}{L}}$,存在情况使得我们的方法在$L$方面表现出指数时间复杂度。
    \end{enumerate}
\end{proposition}
以上命题表明,对于变分量子线路的含噪声期望值模拟,为了在含噪声环境下体现出量子优势,确保噪声率满足$\gamma=o(\frac{1}{\log{L}})$是至关重要的。

在后续的子章节中,将会证明引理~\ref{lemma:cross_items}和命题~\ref{prop:lambda_and_L},以及讨论式~\eqref{eq:generate}减弱的情况。

\subsubsection{引理~\ref{lemma:cross_items}的证明}\label{temp:subsubsection:cross_items}