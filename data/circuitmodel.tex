\chapter{具体线路的复杂度与误差分析}

\section{变分量子线路}
在如今NISQ时代,噪声难以避免的出现在量子线路中。这些噪声导致许多需要高保真度的量子算法难以实现。比如Shor算法需要使用到深层量子线路,因此对噪声的容忍度较低。为了在近期的量子设备中实现实用性的量子算法,变分量子算法作为最受欢迎的NISQ算法,普遍被认为是一个很好的解决方案。变分量子算法的核心思想是通过参数化的量子线路作为
%#TODO

在本节的讨论中,变分量子线路的每一层中的 $\mathcal{U}_i(\bm{\theta}_i)$ 由 $R_i$ 个旋转门和 $C_i$ 个 Clifford 门组成,这些门作用于相互不相交的量子比特上,$\bm{\theta}_i=(\theta_{i,1},\cdots,\theta_{i,R_i})$ 表示第 $i$ 层的参数向量。
具体来说,第 $i$ 层中的第 $j$ 个旋转门表示为 $U_{i, j}(\theta_{i,j})=\exp{-i \frac{\theta_{i,j}}{2} \sigma_{i,j}}$,其中 $j \in \{1, \cdots , R_i\}$,$\theta_{i,j}$ 是变分参数,$\sigma_{i, j}\in \{\mathbb{I}, X,Y,Z\}^{\otimes n}$。
类似地,第 $i$ 层中的第 $k$ 个 Clifford 门表示为 $V_{i,k}$,其中 $k \in \{1, \cdots , C_i\}$。$V_{i,k}\in\{\mathrm{H}(a),\mathrm{S}(a),\mathrm{CNOT}(a,b)\}$,其中 $a,b $ 指的是门作用的量子比特的索引。

所有Pauli旋转门 $U_{i,j}(\theta_{i,j})$ 的 Pauli 算符集合为 $\{\sigma_{i,j}\}$。
我们将集合 $\{\overline{\sigma}_{i,j}\}$ 表示为经过 Clifford 门变换后的所有 $\sigma_{i,j}$,即 $\overline{\sigma}_{i,j}= \mathcal{V}_{L} \cdots \mathcal{V}_{i} \sigma_{i,j} \mathcal{V}_{i}^\dagger \cdots \mathcal{V}_{L}^\dagger$,其中 $\mathcal{V}_{i} = \prod_{k=1}^{C_i} V_{i,k}$ 是对应于第 $i$ 层中所有 Clifford 门的张量积的幺正变换。

为了确保引理~\ref{lemma:MSE_l} 的有效性,我们需要一个容易实现的前提条件:集合 $\{\overline{\sigma}_{i,j}\}$ 可以生成 $\{\mathbb{I}, X, Y, Z\}^{\otimes n}$,其相位为 $\{e^{i\psi}|\psi=0,\frac{\pi}{2},\pi,\frac{3\pi}{2}\}$,其形式化表示为
\begin{equation}\label{eq:generate}
  \langle \{\overline{\sigma}_{i,j}\}\rangle/\left(\langle \{\overline{\sigma}_{i,j}\}\rangle\cap\langle i\mathbb{I}^{\otimes n}\rangle\right)=\{\mathbb{I},X,Y,Z\}^{\otimes n},
\end{equation}
这里 $\langle \{\overline{\sigma}_{i,j}\} \rangle$ 指的是由集合 $\{\overline{\sigma}_{i,j}\}$ 生成的 Pauli 子群,这意味着 $\langle \{\overline{\sigma}_{i,j}\}\rangle$ 中的元素可以表示为 $\{\overline{\sigma}_{i,j}\}$ 中元素的有限乘积。
为了证明这一条件确实容易满足,考虑在电路的最后两层中,每个量子比特上都有一层 $R_X$ 门和一层 $R_Z$ 门作用,那么 $\{X_i, Z_i\}_{i=1,\cdots,n}$ 包含在 $\{\overline{\sigma}_{i,j}\}$ 中。这些 $\{ X_i,Z_i \}_{i=1,\cdots,n}$ 足以生成 $\{\mathbb{I},X,Y,Z\}^{\otimes n}$。事实上,这一充分条件可以进一步减弱,如补充材料~{VIII}~\cite{supplementary} 中所示。


\subsubsection{计算复杂度分析}
\begin{proposition}
    计算 $f$ 中第 $i$ 层项的时间和空间复杂度为 $\order{n}$,其等式如下:
    \begin{equation}\label{eq:i-layer_terms}
    \begin{aligned}
    \Tr{s_i\mathcal{U}_i s_{i-1}\mathcal{U}_i^\dagger}&=\Tr{\left(s_i s_{i-1}\right)\big|_{I_i}}\prod_{k=1}^{C_i}\Tr{\left(s_iV_{i,k} s_{i-1}V_{i,k}^{\dagger}\right)\big|_\g{V_{i,k}}}\prod_{\sigma_{i,j}\in C(i,s_{i-1})}\Tr{\left(s_i s_{i-1}\right)\big|_\g{\sigma_{i,j}}}\\
    &\prod_{\sigma_{i,j'}\in AC(i,s_{i-1})} \Bigg\{ \Tr{\left(s_i s_{i-1}\right)\big|_\g{\sigma_{i,j'}}}\cos{\theta_{i,j'}}- \Tr{\left(is_i \sigma_{i,j'}s_{i-1}\right)\big|_\g{\sigma_{i,j'}}}\sin{\theta_{i,j'}}  \Bigg\}.
    \end{aligned}
    \end{equation}
    我们定义 $g:\{\mathbb{I}, X, Y, Z\}^{\otimes n}\cup\{\mathrm{CNOT}_{a,b},\mathrm{H}_a,\mathrm{S}_a\}\rightarrow2^{\{1,\cdots, n\}}$ 为从幺正算符到其作用的量子比特索引的映射。这里 $2^{\{1,\cdots, n\}}$ 表示 $\{1,\cdots, n\}$ 的所有子集。
    为了简化,我们将 $i$ 层中 $n$ 个量子比特的索引根据所应用门的类型分为三组。这些组分别表示为符号 $\big|_{I_i}$、$\big|_\g{V_{i,k}}$ 和 $\big|_\g{\sigma_{i,j}}$,对应于单位门、Clifford 门和 Pauli 旋转门。此外,集合 $C(i,s_{i-1})$ 和 $AC(i,s_{i-1})$ 分别表示 $\{\sigma_{i,j}\}_{j=1}^{R_i}$ 中与 $s_{i-1}$ 对易和反对易的 Pauli 字集合。
    
    \begin{proof}
    
    在我们的设置中,$\mathcal{U}_i$ 由一系列门 $U_{i,1},\cdots,U_{i,R_i}$ 和 $V_{i,1},\cdots,V_{i,C_i}$ 组成,并且不会在每个量子比特上操作两次。
    Pauli 旋转门 $U_{i,j}(\theta_{i,j})$ 的形式为
    \begin{equation}
      U_{i,j}(\theta_{i,j})=\exp{-i \frac{\theta_{i,j}}{2} \sigma_{i,j}}.
    \end{equation}
    
    Clifford 门 $V_{i,k}$ 可以从 $\{\mathrm{H}(a),\mathrm{S}(a),\mathrm{CNOT}(a,b) \}_ {a\neq b}$ 中选择。
    然后
    \begin{equation}
      \begin{aligned}
        \Tr{s_i\mathcal{U}_i s_{i-1}\mathcal{U}_i^\dagger}=\Tr{\left(s_i s_{i-1}\right)\big|_{I_i}}
        \prod_{k=1}^{C_i}\Tr{\left(s_iV_{i,k} s_{i-1}V_{i,k}^{\dagger}\right)\big|_\g{V_{i,k}}}
        \prod_{j=1}^{R_i}\Tr{\left(s_i U_{i,j}(\theta_{i,j}) s_{i-1} U_{i,j}^\dagger(\theta_{i,j})\right) \big|_\g{\sigma_{i,j}}}.
      \end{aligned}
    \end{equation}
    
    Hermitian 操作 $X$ 的指数定义为 Taylar 展开 $\exp{X}=\sum_{k=0}^\infty\frac{X^k}{k!}$。在计算 Pauli 字的旋转时,任何 Pauli 字的平方都是单位 $\sigma^2=\mathbb{I}$,因此我们有
    \begin{equation}
      \exp{-i \frac{\theta}{2} \sigma}=\sum_{k=0}^\infty\frac{(-i \frac{\theta}{2}\sigma)^k}{k!}=\sum_{k=0}^\infty\frac{(-1)^k (\frac{\theta}{2})^{2k}}{(2k)!} \mathbb{I}- i\frac{(-1)^k (\frac{\theta}{2})^{2k+1}}{(2k+1)!}\sigma=\cos{\frac{\theta}{2}}\mathbb{I}-i \sin{\frac{\theta}{2}}\sigma.
    \end{equation}
    
    因此,根据 $\sigma$ 和另一个 Pauli 字 $\sigma'$ 的交换关系,我们有
    \begin{equation}
    \begin{aligned}
    &\sigma'\exp{-i \frac{\theta}{2} \sigma}=\exp{-i \frac{\theta}{2} \sigma}\sigma',\text{ 如果 }\sigma\text{ 与 }\sigma'\text{ 对易。}\\
    &\sigma'\exp{-i \frac{\theta}{2} \sigma}=\exp{i \frac{\theta}{2} \sigma}\sigma',\text{ 如果 }\sigma\text{ 与 }\sigma'\text{ 反对易。}\\
    \end{aligned}
    \end{equation}
    所以我们可以将 $\{\sigma_{i,j}\}$ 分为两种情况。如果 $\sigma_{i,j}$ 与 $s_{i-1}$ 对易,我们有
    
    \begin{equation}
      \begin{aligned}
      \Tr{\left(s_i U_{i,j}(\theta_{i,j}) s_{i-1} U_{i,j}^\dagger(\theta_{i,j})\right)\big|_\g{\sigma_{i,j}}}
      &=\Tr{(s_i \exp{-i \frac{\theta_{i,j}}{2}\sigma_{i,j} }  \exp{i \frac{\theta_{i,j}}{2}\sigma_{i,j} }s_{i-1}) \big|_\g{\sigma_{i,j}}}\\
      &=\Tr{(s_i \exp{-i \frac{\theta_{i,j}}{2}\sigma_{i,j} } \exp{i \frac{\theta_{i,j}}{2}\sigma_{i,j} }s_{i-1}) \big|_{\g{\sigma_{i,j}}}}\\
      &=\Tr{(s_i s_{i-1})\big|_{\g{\sigma_{i,j}}}}.
    \end{aligned}
    \end{equation}
    而 $\sigma_{i,j}$ 与 $s_{i-1}$ 反对易,我们有
    \begin{equation}
      \begin{aligned}
      \Tr{\left(s_i U_{i,j}(\theta_{i,j}) s_{i-1} U_{i,j}^\dagger(\theta_{i,j})\right)\big|_\g{\sigma_{i,j}}}
      &=\Tr{ \left(s_i\exp{-i \frac{\theta_{i,j}}{2}\sigma_{i,j} } s_{i-1}\exp{i \frac{\theta_{i,j}}{2}\sigma_{i,j}}\right) \big|_\g{\sigma_{i,j}}}\\
      &=\Tr{\left( s_i  \exp{-i\theta_{i,j}\sigma_{i,j}} s_{i-1}\right) \big|_\g{\sigma_{i,j}}}\\
      &=\Tr{ \left(s_i (\cos{\theta_{i,j}}\mathbb{I}-i\sin{\theta_{i,j}}\sigma_{i,j}) s_{i-1} \right) \big|_\g{\sigma_{i,j}}}\\
      &=\Tr{\left(s_i s_{i-1}\right)\big|_\g{\sigma_{i,j}}}\cos{\theta_{i,j}}-\Tr{\left(i s_i \sigma_{i,j}s_{i-1}\right)\big|_\g{\sigma_{i,j}}}\sin{\theta_{i,j}}.
    \end{aligned}
    \end{equation}
    这些完成了 Eq.~\eqref{ap:eq:gate_ele} 的证明。
    \end{proof}
\end{proposition}

\begin{remark}\label{remark:f_ele}
通过利用 Pauli 字的正交性,我们可以建立以下关于 $\Tr{s_i\mathcal{U}_i s_{i-1}\mathcal{U}_i^\dagger} \neq 0$ 的关系:


\begin{tabular}{|c|c|c|}
  \hline
   项 & 关系 & $f$ 中的因子\\
  \hline
  ${I_i}$ &$s_{i-1}\big|_{I_i} = s_{i}\big|_{I_i}$& 1 \\
  \hline
  $V_{i,k}$ &$s_{i-1}\big|_\g{V_{i,k}}= \pm V_{i,k}^{\dagger} s_{i}V_{i,k}\big|_\g{V_{i,k}}$& $\pm 1$ \\
  \hline
  $C(i,s_{i-1})=C(i,s_{i})$&$s_{i}|_\g{\sigma_{i,j}}=s_{i-1}|_\g{\sigma_{i,j}}$& 1 \\
  \hline
  \multirow{2}{*}{$AC(i,s_{i-1})=AC(i,s_{i})$}
  &$s_{i}|_\g{\sigma_{i,j}}=s_{i-1}|_\g{\sigma_{i,j}}$& $\cos{\theta_{i,j}}$ \\
  \cline{2-3}
  &$s_{i}|_\g{\sigma_{i,j}}=\pm i \sigma_{i,j} s_{i-1}|_\g{\sigma_{i,j}}$& $\mp \sin{\theta_{i,j}}$ \\
  \hline
\end{tabular}
\end{remark}

