\chapter{Clifford扰动量子线路的模拟}


\section{背景介绍}
在过去的章节中,我们介绍了使用Stabilizer稳定子方法实现由Clifford门组成Clifford量子线路的高效模拟。
在基于稳定子的模拟方法中,只能高效模拟Clifford量子线路,或是只有少量的非Clifford门的由Clifford门主导的量子线路。
但对于存在大量的非Clifford门的量子线路,即使这些Clifford门非常接近Clifford门,基于Stabilizer稳定子的方法也很难进行进行高效模拟。

在这一章中,我们将讨论如何利用SPD模拟算法模拟Clifford扰动量子线路。Clifford扰动量子线路是由接近Clifford门的门组成的量子线路。
Clifford扰动量子线路的模拟可以用于量子错误缓解等领域,通过扩展Clifford数据回归方法(Clifford Data Regression,CDR)将错误缓解的回归数据集扩展到Clifford扰动量子线路上,可以显著的提高错误缓解的效果~\cite{zhang2024clifford},这种方法被称为Clifford扰动数据回归(Clifford Perturbation Data Regression,CPDR)。

%类似于第~\ref{chap:noisy_vqa}章,
在我们的线路模型中Clifford扰动线路是一类特殊的变分量子线路,对于一个$L$层的Clifford扰动量子线路,我们可以表示为
\begin{equation}\label{eq:parameterized_circuit}
    \mathcal{U}(\bm{\theta})=\mathcal{U}_L(\theta_L)  \cdots \mathcal{U}_1(\theta_1),
\end{equation}
其中$\bm{\theta}=(\theta_1,\cdots,\theta_L)$是旋转角度向量。每一个酉算子$\mathcal{U}_i(\theta_i):=\exp{-i \theta_i P_i / 2}C_i $ 包含一个Clifford算符$C_i$和一个在Pauli算符 $P_i\in\{\mathbb{I},X,Y,Z\}^{\otimes n}$上旋转$\theta_i$角度的Pauli旋转算符$\exp{-i \theta_i P_i / 2}$。
注意到,当旋转角度$\theta$取值为$\{\frac{k\pi}{4}\mid k\in \mathbb{Z}\}$时,Pauli旋转门$e^{-i\frac{\theta}{2} P}=\cos{\frac{\theta}{2}}\mathbb{I}+i\sin{\frac{\theta}{2}}P$属于Clifford群。

因此,我们可以假设在方程~\eqref{eq:parameterized_circuit}中的旋转角度$\theta_i$位于$[-\frac{\pi}{4},\frac{\pi}{4}]$范围内。对于任何超出此范围的$\theta_i$,存在$\theta'_i\in [-\frac{\pi}{4},\frac{\pi}{4}]$使得$\theta'_i+\frac{k\pi}{2}=\theta_i$。
此时,酉算子$\mathcal{U}_i(\theta_i)$可以写成$\mathcal{U}_i(\theta_i)=e^{-i\frac{\theta}{2} P_i}C_i=e^{-i\frac{\theta'_i}{2} P_i}e^{-i\frac{k\pi}{4} P_i}C_i$,其中$e^{-i\frac{k\pi}{4} P_i}C_i$是一个Clifford算符。通过将$C_i$替换为$C_i'=e^{-i\frac{k\pi}{4} P_i}C_i$,酉算子变为$\mathcal{U}_i(\theta_i)=e^{-i\frac{\theta'_i}{2} P_i}C_i'$,其中$\theta'_i\in [-\frac{\pi}{4},\frac{\pi}{4}]$。
由于旋转角度被限制在较窄的范围$[-\frac{\pi}{4},\frac{\pi}{4}]$内,这些Pauli旋转算符,可以被解释为对Clifford算符的扰动。

在我们的讨论中,量子线路$\mathcal{U}(\bm{\theta})$作用到初始量子态$\rho$上,目标是近似演化后的量子态在可观测量$O$下的观测期望值:
\begin{equation}
    \langle O \rangle = \tr{O \mathcal{U}(\bm{\theta})\rho \mathcal{U}(\bm{\theta})^\dagger}.
\end{equation}
在本章中,如不加说明,假设可观测量$O$是一个Pauli算符,即$O\in \{\mathbb{I},X,Y,Z\}^{\otimes n}$。
通过使用SPD算法截断高阶扰动项,可以高效地近似观测期望值。
首先注意到,Pauli旋转门$e^{i\frac{\theta}{2} P}$作用在Pauli算符$\sigma$上的演化描述为:
\begin{equation}\label{eq:heisenberg}
  e^{i \frac{\theta}{2} P} \sigma e^{-i \frac{\theta}{2} P} = 
  \begin{cases}
  \sigma, & [\sigma,P] = 0, \\
  \cos(\theta) \sigma + i \sin(\theta) P\sigma & \{\sigma, P\} = 0.
  \end{cases} 
\end{equation}
对于Pauli算符的可观测量$O$,在SPD算法的反向传递过程中,首先会初始化$s_L=O$,然后通过对每个酉算子$\mathcal{U}_i(\theta_i)$迭代。
在反向传递过程的第一步通过$\mathcal{U}_L$时,有${U}_L(\theta_L)^{\dagger} O {U}_L(\theta_L)=C^\dagger_Le^{-i\frac{\theta}{2} P_L}Oe^{-i\frac{\theta}{2} P_L}C_L$在$[O,P_L]=0$时简化为$C_L^\dagger O C_L$,此时$s_{L-1}=C_L^\dagger O C_L$是唯一的可能。
相反,对于$\{O,P_L\} = 0$,它变为$\cos(\theta_L)C_L^\dagger O C_L + i\sin(\theta_L) C_L^\dagger P_L O C_L$。此时,$s_{L-1}$发生分叉,有两种可能的值$s_{L-1}^{(0)}=C_L^\dagger O C_L$和$s_{L-1}^{(1)}=C_L^\dagger P_L O C_L$。
因为,$\abs{\Tr{s_L \mathcal{U}_{L} s_{L-1}^{(0)} \mathcal{U}_{L}^\dagger}}=\abs{\cos(\theta_L)}$,而$\abs{\Tr{s_L \mathcal{U}_{L} s_{L-1}^{(1)} \mathcal{U}_{L}^\dagger}}=\abs{\sin(\theta_L)}$,根据$\theta_i\in [-\frac{\pi}{4},\frac{\pi}{4}]$总是有$\abs{\cos(\theta_i)}\geq \abs{\sin(\theta_i)}$,因此我们总是有$h(s_{L-1}^{(0)},s_L,\mathcal{U}_{L})=0$和$h(s_{L-1}^{(1)},s_L,\mathcal{U}_{L})=1$,
其中$h(s_{i-1},s_i,U_{i,j})$是一个二值函数,定义于式~\eqref{eq:spd_h}。
因此,在Clifford扰动线路的情况中,由式~\eqref{eq:spd_g}中定义的对每个Pauli路径$\bm{s}$定义的函数$g$:
\begin{equation}
    g(\bm{s})=\sum_{i=1}^{L}\sum_{j} h(s_{i-1},s_i,U_{i,j}),
\end{equation}
其实是该Pauli路径对应的,定义于式~\eqref{eq:pp:contribution}中,贡献函数$f(\mathcal{U},\bm{s},O,\rho)$里的$\sin$因子的数量。

因为在SPD算法中,近似的观测期望值是枚举所有满足$\{\bm{s}\mid g(\bm{s})\leq M\}$的Pauli路径,并通过式~\eqref{eq:spd:approximation}计算的,所以我们可以直接通过截断$\sin$因子的个数来近似观测期望值。
具体来说,在海森堡图景中~\cite{zhang2024clifford},观测期望值被重新表述为$\langle O \rangle = \tr{\rho \tilde{O}}$,其中$\tilde{O} = \mathcal{U}_1(\theta_1)^{\dagger} \cdots  \mathcal{U}_L(\theta_L)^{\dagger} O \mathcal{U}_L(\theta_L)  \cdots \mathcal{U}_1(\theta_1)$表示海森堡演化后的可观测量。
通过对所有酉算子$U_i(\theta_i)$迭代这个过程,根据式~\eqref{eq:heisenberg},得到的海森堡演化后的可观测量$\tilde{O}$呈现为Pauli算符的线性组合,表示为$\tilde{O} = \sum_\sigma c_\sigma \sigma$,其中$c_\sigma$是由$\cos(\theta_i)$和$\sin(\theta_i)$乘积组成的多变量三角多项式。

为了将扰动项截断到阶数$M$,如果$c_\sigma$中的$\sin$因子的数量超过$M$,我们丢弃$c_\sigma \sigma$项。截断后的可观测量表示为$\tilde{O}_M=\sum_{\{\sigma\mid \abs{c_\sigma}_{\sin}\leq M\}} c_\sigma \sigma$,其中$\abs{c_\sigma}_{\sin}$表示$c_\sigma$中$\sin$因子的数量。
近似的观测期望值表示为:
\begin{equation}\label{eq:sim_noiseless}
  \langle O \rangle ^{(M)}= \tr{\rho \tilde{O}_M}= \sum_{\bm{s}\in \{\bm{s}\mid g(\bm{s})\leq M\}} f(\mathcal{U},\bm{s},O,\rho).
\end{equation}

接下来,我们将讨论使用SPD算法计算$\langle O \rangle ^{(M)}$的计算复杂度和误差估计。

\section{计算复杂度分析}
根据式~\eqref{eq:spd_cost},我们可以看到,计算$\langle O \rangle ^{(M)}$的计算复杂度可以表示为$C = C_s+N_MC_f$,其中$C_s$是枚举所有满足条件的Pauli路径的时间复杂度,$N_M$是满足条件的Pauli路径的数量,$C_f$是计算每个Pauli路径的贡献的时间复杂度。
因为Clifford扰动量子线路仍然属于变分量子线路,与第~\ref{chap:noisy_vqa}章类似,根据式~\eqref{eq:vqa:c_f},可以得到:
\begin{equation}
    C_f=\order{L n}+\order{n}+\order{\poly(n)}=\order{\poly(n)+L n}.
\end{equation}
根据式~\eqref{eq:spd:cost_s},对于SPD算法,我们有
\begin{equation}
    C_s=\mathrm{Poly}(n) \order{(nL)^{M}} C_{\text{split}} N_{\text{max}}^{M},
\end{equation}
其中$C_{\text{split}}$是分裂操作的时间复杂度,$N_{\text{max}}$是每个门的最大可分裂的Pauli算符的数量。与之前含噪变分量子线路模拟的情况类似,我们有$C_{\text{split}}=\order{n}$和$N_{\text{max}}=2$。
根据式~\eqref{eq:spd:N_M}满足条件的Pauli路径的数量$N_M$最多为:
\begin{equation}
    N_M=\mathrm{Poly}(n) (nLN_{\text{max}})^{M+1}.
\end{equation}
因此,我们可以得到计算$\langle O \rangle ^{(M)}$的总体计算复杂度为:
\begin{equation}
    C=\order{\mathrm{Poly}(n) (2nL)^{M+1}}.
\end{equation}

对于空间复杂度,根据式~\eqref{eq:spd_space},可以表示为$S = S_s+S_f$,其中$S_s$是枚举Pauli路径过程的空间复杂度,$S_f$是计算每个Pauli路径的贡献的空间复杂度。
对于枚举过程的空间复杂度,根据式~\eqref{eq:spd:space_cost}有:
\begin{equation}
    S_{s}=\order{\mathrm{Poly}(n)+\max\{S_{\text{split}},n\}L}=\order{\mathrm{Poly}(n)+nL}.
\end{equation}
对于计算每个Pauli路径的贡献的空间复杂度,根据式~\eqref{eq:space_f}有:
\begin{equation}
    S_f=\max\{\order{n}+\order{\poly(n)}+\order{\poly(n)}\}=\order{\poly(n)}.
\end{equation}


综上所述,对于Clifford扰动量子线路,在给定截断参数$M$的情况下,使用SPD算法计算$\langle O \rangle ^{(M)}$的时间复杂度$C$和空间复杂度$S$分别为:
\begin{equation}\label{eq:cpc:cost}
    \begin{aligned}
        C&=\mathrm{Poly}(n) \order{(2nL)^{M+1}},\\
        S&=\mathrm{Poly}(n)+nL,
    \end{aligned}
\end{equation}
其中$n$为量子比特数,$L$为线路深度。

\section{误差分析}
当量子线路中的旋转角度较小时,则$\sin{\theta}$项的值较小,因此对$\sin{\theta}$项的截断将提供对期望值的良好近似。
为了证明这一点,我们引入小角度空间$\Theta=\{\bm{\theta}\mid \abs{\theta_i}\leq \theta_* ,i=1,\cdots,L \}$,并分析小角度空间中的截断误差。



首先,我们需要利用以下引理来证明在小角度空间中,不同Pauli路径的贡献函数相互正交:
\begin{lemma}
对于任意不同的Pauli路径$\bm{s},\bm{s}^{\prime}\in \bm{P}^{L+1}_n$,以及小角度空间$\Theta=\{\bm{\theta}\mid \abs{\theta_i}\leq \theta_* ,i=1,\cdots,L \}$,我们有
\begin{equation}\label{ap:eq:E_cross_equals_0}
\mathbb{E}_{\Theta}f(\mathcal{U},\bm{s},O,\rho)f(\mathcal{U},\bm{s}^{\prime},O,\rho)=0.
\end{equation}
\end{lemma}

\begin{proof}
    这个引理可以通过分析演化过程来证明。根据式~\eqref{eq:pp:contribution},我们有
    \begin{equation}
        \begin{aligned}
            f(\mathcal{U},\bm{s},O,\rho)f(\mathcal{U},\bm{s}^{\prime},O,\rho)=&\Tr{Os_L}\Tr{Os'_L}\\
            &\left(\prod_{i=1}^{L}\Tr{s_i\mathcal{U}_i(\theta_i) s_{i-1}\mathcal{U}_i(\theta_i)^\dagger}\Tr{s'_i\mathcal{U}_i(\theta_i) s'_{i-1}\mathcal{U}_i(\theta_i)^\dagger}\right)\\
            &\Tr{s_0 \rho}\Tr{s'_0 \rho}
        \end{aligned}
    \end{equation}
    首先,必须有$s_L=s'_L$,因为初始可观测量是相同的Pauli算符,否则因子$\Tr{Os_L}\Tr{Os'_L}$为零。
    然后,我们考虑在算子$\mathcal{U}_L(\theta_L)=C_L\exp{-i \theta_i P_L / 2}$下$s_{L-1}$和$s'_{L-1}$的演化。

    如果$P_L$与$C_L^\dagger s_L C_L$对易,根据方程~\eqref{eq:heisenberg},我们有$\mathcal{U}_L(\theta_L)^\dagger s_L \mathcal{U}_L(\theta_L)=\mathcal{U}_L(\theta_L)^\dagger s'_L \mathcal{U}_L(\theta_L) = C_L^\dagger s_L C_L$,那么$s_{L-1}=s'_{L-1}$,否则因子$\Tr{s_L\mathcal{U}_L(\theta_L) s_{L-1}\mathcal{U}_L(\theta_L)^\dagger}\Tr{s'_L\mathcal{U}_L(\theta_L) s'_{L-1}\mathcal{U}_L(\theta_L)^\dagger}$为零。

    如果$P_L$与$C_L^\dagger s_L C_L$反对易,根据方程~\eqref{eq:heisenberg},$s_{L-1}$和$s'_{L-1}$可以是$C_L^\dagger s_L C_L$或$P_L C_L^\dagger s_L C_L$,否则$\Tr{s_L\mathcal{U}_L(\theta_L) s_{L-1}\mathcal{U}_L(\theta_L)^\dagger}\Tr{s'_L\mathcal{U}_L(\theta_L) s'_{L-1}\mathcal{U}_L(\theta_L)^\dagger}=0$。如果$s_{L-1}\neq s'_{L-1}$,意味着在乘积$f(\mathcal{U},\bm{s},O,\rho)f(\mathcal{U},\bm{s}^{\prime},O,\rho)$中存在因子$\cos{\theta_L}\sin{\theta_L}$,由于
    \begin{equation}
        \int_{-\theta_*}^{\theta_*}\cos{\theta_L}\sin{\theta_L}\mathrm{d}\theta_L=0,
    \end{equation}
    因此我们必须有$s_{L-1}=s'_{L-1}$。重复这个过程,我们有$s=s'$,否则$\mathbb{E}_{\Theta}f(\mathcal{U},\bm{s},O,\rho)f(\mathcal{U},\bm{s}^{\prime},O,\rho)=0$,这完成了证明。
\end{proof}



对于$\langle O \rangle$和$\langle O \rangle ^{(M)}$之间在平均情况下的误差,我们可以用均方误差来度量,对均方误差我们有如下引理可以提供上界:
\begin{lemma}
    如果可观测量$O$是一个Pauli算符,那么在小角度空间$\Theta$中,$\langle O \rangle$和$\langle O \rangle ^{(M)}$之间的均方误差上界为
    \begin{equation}
        \mathbb{E}_{\Theta}\left(\langle O \rangle - \langle O \rangle ^{(M)}\right)^2 \leq  \left(1+c\right)^L - \left(1+c\right)^M,
    \end{equation}
    其中$c=\frac{\theta_*^2}{3}+\order{\theta_*^4}$是一个常数。
\end{lemma}

\begin{proof}
    根据式~\eqref{eq:sim_noiseless},近似期望值的定义,我们有
    \begin{equation}
        \begin{aligned}
        \mathbb{E}_{\Theta}\left(\langle O \rangle - \langle O \rangle ^{(M)}\right)^2 &= \mathbb{E}_{\Theta}\left(\sum_{\bm{s}\in \{\bm{s}\mid g(\bm{s})> M\}} f(\mathcal{U},\bm{s},O,\rho)\right)^2\\
        &=\sum_{s,s'\in\{\bm{s}\mid g(\bm{s})> M\}} \mathbb{E}_{\Theta}\left(f(\mathcal{U},\bm{s},O,\rho)f(\mathcal{U},\bm{s}^\prime,O,\rho) \right)\\
        &=\sum_{\bm{s}\in \{\bm{s}\mid g(\bm{s}) > M\}} \mathbb{E}_{\Theta}\left(f(\mathcal{U},\bm{s},O,\rho)\right)^2,
      \end{aligned}
    \end{equation}
    其中最后一个等式由方程~\eqref{ap:eq:E_cross_equals_0}得出。

    根据Hölder不等式,在对于$p,q>0$满足$\frac{1}{p}+\frac{1}{q}=1$时,有$\abs{\tr{A^\dagger B}}\leq \norm{A}_p\norm{B}_q$,我们有
    \begin{equation}
        \begin{aligned}
            f(\mathcal{U},\bm{s},O,\rho)^2=&\Tr{Os_L}^2\tr{\rho s_0}^2\left(\prod_{i=1}^{L}\Tr{s_i\mathcal{U}_i(\theta_i) s_{i-1}\mathcal{U}_i(\theta_i)^\dagger}\right)^2\\
        \leq & \left(\norm{O}_2\norm{s_L}_2 \norm{\rho}_1\norm{s_0}_\infty\right)^2  \left(\prod_{i=1}^{L}\Tr{s_i\mathcal{U}_i(\theta_i) s_{i-1}\mathcal{U}_i(\theta_i)^\dagger}\right)^2\\
        =& \left(\prod_{i=1}^{L}\Tr{s_i\mathcal{U}_i(\theta_i) s_{i-1}\mathcal{U}_i(\theta_i)^\dagger}\right)^2,
        \end{aligned}
    \end{equation}
    其中最后一个等式成立是因为$\norm{O}_2=\sqrt{\tr{OO^\dagger}}=\sqrt{2^n}$,$\norm{\rho}_1=\tr{\rho}=1$,$\norm{s_L}_2=\sqrt{\tr{s_L^\dagger s_L}}=1$,$\norm{s_0}_\infty=\left(\frac{1}{\sqrt{2}}\right)^2$。

    集合$\{\bm{s}\mid g(\bm{s})=l\}$表示在其贡献中拥有正好$l$个$\sin{\theta}$项的Pauli路径集合。因为总共有$L$个门至多$L$个分叉点,于是集合$\{\bm{s}\mid g(\bm{s})= l\}$的数量上界为$\binom{L}{l}$。

    如果我们将常数$c$记为:
    \begin{equation}
        c=\mathbb{E}_{\abs{\theta}\leq \theta_*}\sin^2{\theta}=\frac{1}{2\theta_*} \int_{-\theta_*}^{\theta_*}(\sin^2{\theta})\mathrm{d}\theta= \frac{1}{2}-\frac{\sin{2\theta_*}}{2\theta_*}=\frac{ \theta_*^2}{3}+\order{\theta_*^4}.
    \end{equation}

    那么对于任何$s\in\{\bm{s}\mid g(\bm{s})= l\}$,我们有
    \begin{equation}
        \begin{aligned}
        \mathbb{E}_{\Theta}f(\mathcal{U},\bm{s},O,\rho)^2 &\leq \mathbb{E}_{\Theta}\left(\prod_{i=1}^{L}\Tr{s_i\mathcal{U}_i(\theta_i) s_{i-1}\mathcal{U}_i(\theta_i)^\dagger}\right)^2\\
        &= \prod_{i=1}^{L}\mathbb{E}_{\abs{\theta_i}\leq \theta_*}\Tr{s_i\mathcal{U}_i(\theta_i) s_{i-1}\mathcal{U}_i(\theta_i)^\dagger}^2\\
        & \leq c^{l},
        \end{aligned}
    \end{equation}
    其中不等式成立是因为$\Tr{s_i\mathcal{U}_i(\theta_i) s_{i-1}\mathcal{U}_i(\theta_i)^\dagger}^2$等于$\cos^2{\theta_i}$或$\sin^2{\theta_i}$或$1$,并且$\mathbb{E}_{\abs{\theta_i}\leq \theta_*}\cos^2{\theta_i}\leq 1$和$\mathbb{E}_{\abs{\theta_i}\leq \theta_*}\sin^2{\theta_i}=c$。

    因此,我们有
    \begin{equation}
    \begin{aligned}
        \sum_{s\in\{\bm{s}\mid g(\bm{s})>M\}} \mathbb{E}_{\Theta}f(\mathcal{U},\bm{s},O,\rho)^2 &\leq \sum_{l=M+1}^{L} \sum_{\{\bm{s}\mid g(\bm{s})=l\}} \mathbb{E}_{\Theta}f(\mathcal{U},\bm{s},O,\rho)^2\\
        & \leq \sum_{l=M+1}^{L} \sum_{\{\bm{s}\mid g(\bm{s})=l\}} c^l \\
        & \leq \sum_{l=M+1}^{L} \binom{L}{l} c^l\\
        & \leq  \left(1+c\right)^L-\left(1+c\right)^M.\\
      \end{aligned}
\end{equation}

\end{proof}

对于最差情况下的误差,我们有如下引理:
\begin{lemma}
    对于Pauli可观测量$O$,在小角度空间$\Theta$中,$\langle O \rangle$和$\langle O \rangle ^{(M)}$之间的差异上界为
    \begin{equation}
        \abs{\langle O \rangle - \langle O \rangle ^{(M)}} \leq  \left(1+\sin{\theta_*}\right)^L - \left(1+\sin{\theta_*}\right)^M.
    \end{equation}
\end{lemma}

\begin{proof}
    根据式~\eqref{eq:sim_noiseless},近似期望值的定义,我们有
    \begin{equation}
        \begin{aligned}
            \abs{\langle O \rangle - \langle O \rangle ^{(M)}} &= \abs{\sum_{s\in\{\bm{s}\mid g(\bm{s})>M\}} f(s,\bm{\theta}) \tr{\rho s_0}} \\
            &\leq \sum_{s\in\{\bm{s}\mid g(\bm{s})>M\}} \abs{f(s,\bm{\theta}) \tr{\rho s_0}}\\
            &= \sum_{l=M+1}^{L} \sum_{\{\bm{s}\mid g(\bm{s})=l\}} \abs{f(s,\bm{\theta}) \tr{\rho s_0}}\\
            &\leq \sum_{l=M+1}^{L} \sum_{\{\bm{s}\mid g(\bm{s})=l\}} \sin^l{\theta_*} \\
            &\leq \sum_{l=M+1}^{L} \binom{L}{l} \sin^l{\theta_*}\\
            &\leq  \left(1+\sin{\theta_*}\right)^L - \left(1+\sin{\theta_*}\right)^M.
        \end{aligned}
    \end{equation}
    这完成了证明。
\end{proof}

注意到,对于任意$\delta>0$,如果$0<c<\frac{\ln{1+\frac{\delta}{2}}}{L-M}\leq \frac{\ln{2}}{M}$,那么我们有:
\begin{equation}
  \begin{aligned}
     \left(1+c\right)^{L}-\left(1+c\right)^{M}&= \left(1+c\right)^{M} \left(\left(1+c\right)^{L-M}-1 \right)\\
    &\leq 2\left(1+\frac{\delta}{2}-1\right)=\delta.
  \end{aligned}
\end{equation}
其中最后一个不等式是由于$(1+x)^m\leq e^{mx}$。因此$(1+c)^M\leq e^{cM}\leq 2$且$\left(1+c\right)^{L-M}\leq e^{c(L-M)}\leq 1+\frac{\delta}{2}$。

综上所述,我们有以下定理:
\begin{theorem}
    对于任意给定的$\delta>0$,如果$\frac{\ln{1+\frac{\delta}{2}}}{L-M}\leq \frac{\ln{2}}{M}$,且$\theta_*=\frac{\ln{1+\frac{\delta}{2}}}{L-M}=\order{\frac{1}{L}}$,则有截断误差$\abs{\langle O \rangle - \langle O \rangle ^{(M)}}\leq \delta$。在平均情况下,如果有$\theta_*=\sqrt{\frac{3\ln{1+\frac{\delta}{2}}}{L-M}}=\order{\frac{1}{\sqrt{L}}}$,则有均方误差为了使均方误差小于$\mathbb{E}_{\Theta}\left(\langle O \rangle - \langle O \rangle ^{(M)}\right)^2 \leq \delta$。
\end{theorem}

可以看到在定理的描述中,$M$的取值似乎对于小角度空间的刻画$\theta_*$并不敏感。并且假设要实现$10^{-2}$的截断误差,根据解析结果需要$\theta_*$大概为$\ln 1+\frac{10^{-2}}{2} \approx 0.005$,这是一个非常小的角度。但在实际应用中,可以看到这个解析结果是非常宽松的,实际可用的角度远大于这个值。并且,稍大的$M$即可保证截断误差在可接受范围内。我们将在后续的数值结果中详细讨论。

\section{本章小结}
在本章中,我们讨论了Clifford扰动量子线路的模拟问题。
我们首先介绍了Clifford扰动量子线路的定义,与Clifford主导的量子线路不同,Clifford扰动量子线路是包含了大量接近Clifford的非Clifford门,因此难以被Stabilizer稳定子模拟起直接模拟。

然后,我们介绍了SPD算法,并将其应用于Clifford扰动量子线路的模拟问题。我们分析了SPD算法的计算复杂度和空间复杂度,并讨论了截断误差的上界。最后,我们证明了在小角度空间中,SPD算法可以提供高效的近似期望值,并且可以保证截断误差和均方误差在可接受范围内。在下一章中,我们将通过数值实验验证我们的理论结果。