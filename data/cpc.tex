\chapter{Clifford扰动量子线路的模拟}

在过去的章节中,我们介绍了使用Stabilizer稳定子方法实现由Clifford门组成Clifford量子线路的高效模拟。在这一章中,我们将讨论如何利用SPD模拟算法模拟Clifford扰动量子线路。Clifford扰动量子线路是由接近Clifford门的门组成的量子线路。
Clifford扰动量子线路的模拟可以用于量子错误缓解等领域,通过扩展Clifford数据回归方法(Clifford Data Regression,CDR)将错误缓解的回归数据集扩展到Clifford扰动量子线路上,可以显著的提高错误缓解的效果~\cite{zhang2024clifford},这种方法被称为Clifford扰动数据回归(Clifford Perturbation Data Regression,CPDR)。

类似于第~\ref{chap:noisy_vqa}章,在我们的线路模型中Clifford扰动线路是一类特殊的变分量子线路,对于一个$L$层的Clifford扰动量子线路,我们可以表示为
\begin{equation}\label{eq:parameterized_circuit}
    \mathcal{U}(\bm{\theta})=\mathcal{U}_L(\theta_L)  \cdots \mathcal{U}_1(\theta_1),
\end{equation}
其中$\bm{\theta}=(\theta_1,\cdots,\theta_L)$是旋转角度向量。每一个酉算子$\mathcal{U}_i(\theta_i):=\exp{-i \theta_i P_i / 2}C_i $ 包含一个Clifford算符$C_i$和一个在Pauli算符 $P_i\in\{\mathbb{I},X,Y,Z\}^{\otimes n}$上旋转$\theta_i$角度的Pauli旋转算符$\exp{-i \theta_i P_i / 2}$。
注意到,当旋转角度$\theta$取值为$\{\frac{k\pi}{4}\mid k\in \mathbb{Z}\}$时,Pauli旋转门$e^{-i\frac{\theta}{2} P}=\cos{\frac{\theta}{2}}\mathbb{I}+i\sin{\frac{\theta}{2}}P$属于Clifford群。
因此,我们可以假设在方程~\eqref{eq:parameterized_circuit}中的旋转角度$\theta_i$位于$[-\frac{\pi}{4},\frac{\pi}{4}]$范围内。对于任何超出此范围的$\theta_i$,存在$\theta'_i\in [-\frac{\pi}{4},\frac{\pi}{4}]$使得$\theta'_i+\frac{k\pi}{2}=\theta_i$。在这种情况下,酉算子$\mathcal{U}_i(\theta_i)$可以写成$\mathcal{U}_i(\theta_i)=e^{-i\frac{\theta}{2} P_i}C_i=e^{-i\frac{\theta'_i}{2} P_i}e^{-i\frac{k\pi}{4} P_i}C_i$,其中$e^{-i\frac{k\pi}{4} P_i}C_i$是一个Clifford算符。通过将$C_i$替换为$C_i'=e^{-i\frac{k\pi}{4} P_i}C_i$,酉算子变为$\mathcal{U}_i(\theta_i)=e^{-i\frac{\theta'_i}{2} P_i}C_i'$,其中$\theta'_i\in [-\frac{\pi}{4},\frac{\pi}{4}]$。
由于旋转角度被限制在较窄的范围$[-\frac{\pi}{4},\frac{\pi}{4}]$内,Pauli旋转可以解释为Clifford算符的扰动。

在我们的讨论中,量子线路$\mathcal{U}(\bm{\theta})$作用到初始量子态$\rho$上,目标是近似演化后的量子态在可观测量$O$下的观测期望值:
\begin{equation}
    \langle O \rangle = \tr{O \mathcal{U}(\bm{\theta})\rho \mathcal{U}(\bm{\theta})^\dagger}.
\end{equation}
通过使用SPD算法截断高阶扰动项,可以高效地近似观测期望值。
首先注意到,Pauli旋转门$e^{i\frac{\theta}{2} P}$作用在Pauli算符$\sigma$上的演化描述为:
\begin{equation}\label{eq:heisenberg}
  e^{i \frac{\theta}{2} P} \sigma e^{-i \frac{\theta}{2} P} = 
  \begin{cases}
  \sigma, & [\sigma,P] = 0, \\
  \cos(\theta) \sigma + i \sin(\theta) P\sigma & \{\sigma, P\} = 0.
  \end{cases} 
\end{equation}
对于Pauli算符的可观测量$O$,在SPD算法的反向传递过程中,首先会初始化$s_L=O$,然后通过对每个酉算子$\mathcal{U}_i(\theta_i)$迭代。
在反向传递过程的第一步通过$\mathcal{U}_L$时,有${U}_L(\theta_L)^{\dagger} O {U}_L(\theta_L)=C^\dagger_Le^{-i\frac{\theta}{2} P_L}Oe^{-i\frac{\theta}{2} P_L}C_L$在$[O,P_L]=0$时简化为$C_L^\dagger O C_L$,此时$s_{L-1}=C_L^\dagger O C_L$是唯一的可能。
相反,对于$\{O,P_L\} = 0$,它变为$\cos(\theta_L)C_L^\dagger O C_L + i\sin(\theta_L) C_L^\dagger P_L O C_L$。此时,$s_{L-1}$发生分叉,有两种可能的值$s_{L-1}^{(0)}=C_L^\dagger O C_L$和$s_{L-1}^{(1)}=C_L^\dagger P_L O C_L$。
因为,$\abs{\Tr{s_L \mathcal{U}_{L} s_{L-1}^{(0)} \mathcal{U}_{L}^\dagger}}=\abs{\cos(\theta_L)}$,而$\abs{\Tr{s_L \mathcal{U}_{L} s_{L-1}^{(1)} \mathcal{U}_{L}^\dagger}}=\abs{\sin(\theta_L)}$,根据$\theta_i\in [-\frac{\pi}{4},\frac{\pi}{4}]$总是有$\abs{\cos(\theta_i)}\geq \abs{\sin(\theta_i)}$,因此我们总是有$h(s_{L-1}^{(0)},s_L,\mathcal{U}_{L})=0$和$h(s_{L-1}^{(1)},s_L,\mathcal{U}_{L})=1$,
其中$h(s_{i-1},s_i,U_{i,j})$是一个二值函数,定义于式~\eqref{eq:spd_h}。
因此,在Clifford扰动线路的情况中,由式~\eqref{eq:spd_g}中定义的对每个Pauli路径$\bm{s}$定义的函数$g$:
\begin{equation}
    g(\bm{s})=\sum_{i=1}^{L}\sum_{j} h(s_{i-1},s_i,U_{i,j}),
\end{equation}
其实是该Pauli路径对应的,定义于式~\eqref{eq:pp:contribution}中,贡献函数$f(\mathcal{U},\bm{s},O,\rho)$里的$\sin$因子的数量。
因为,在SPD算法中,近似的观测期望值是枚举所有满足$\{\bm{s}\mid g(\bm{s})\leq M\}$的Pauli路径,并通过式~\eqref{eq:spd:approximation}计算的,所以我们可以直接通过截断$\sin$因子的个数来近似观测期望值。

具体来说,在海森堡图景中~\cite{zhang2024clifford},观测期望值被重新表述为$\langle O \rangle = \tr{\rho \tilde{O}}$,其中$\tilde{O} = \mathcal{U}_1(\theta_1)^{\dagger} \cdots  \mathcal{U}_L(\theta_L)^{\dagger} O \mathcal{U}_L(\theta_L)  \cdots \mathcal{U}_1(\theta_1)$表示海森堡演化后的可观测量。
通过对所有酉算子$U_i(\theta_i)$迭代这个过程,根据式~\eqref{eq:heisenberg},得到的海森堡演化后的可观测量$\tilde{O}$呈现为Pauli算符的线性组合,表示为$\tilde{O} = \sum_\sigma c_\sigma \sigma$,其中$c_\sigma$是由$\cos(\theta_i)$和$\sin(\theta_i)$乘积组成的多变量三角多项式。

为了将扰动项截断到阶数$M$,如果$c_\sigma$中的$\sin$因子的数量超过$M$,我们丢弃$c_\sigma \sigma$项。截断后的可观测量表示为$\tilde{O}_M=\sum_{\{\sigma\mid \abs{c_\sigma}_{\sin}\leq M\}} c_\sigma \sigma$,其中$\abs{c_\sigma}_{\sin}$表示$c_\sigma$中$\sin$因子的数量。
近似的观测期望值表示为:
\begin{equation}\label{eq:sim_noiseless}
  \langle O \rangle ^{(M)}= \tr{\rho \tilde{O}_M}= \sum_{\bm{s}\in \{\bm{s}\mid g(\bm{s})\leq M\}} f(\mathcal{U},\bm{s},O,\rho).
\end{equation}

接下来,我们将讨论使用SPD算法计算$\langle O \rangle ^{(M)}$的计算复杂度,和误差估计。

\section{计算复杂度分析}
根据式~\eqref{eq:spd_cost},我们可以看到,计算$\langle O \rangle ^{(M)}$的计算复杂度可以表示为$C = C_s+N_MC_f$,其中$C_s$是枚举所有具有非平凡贡献函数的Pauli路径的时间复杂度,$N_M$是满足条件的Pauli路径的数量,$C_f$是计算每个Pauli路径的贡献的时间复杂度。
因为Clifford扰动量子线路仍然属于变分量子线路,与第~\ref{chap:noisy_vqa}章类似,根据式~\eqref{eq:vqa:c_f},可以得到:
\begin{equation}
    C_f=\order{L n}+\order{n}+\order{\poly(n)}=\order{\poly(n)+L n}.
\end{equation}
根据式~\eqref{eq:spd:cost_s},对于SPD算法,我们有
\begin{equation}
    C_s=\mathrm{Poly}(n) \order{(nL)^{M}} C_{\text{split}} N_{\text{max}}^{M},
\end{equation}
其中$C_{\text{split}}$是分裂操作的时间复杂度,$N_{\text{max}}$是每个门的最大可分裂的Pauli算符的数量。与之前含噪变分量子线路模拟的情况类似,我们有$C_{\text{split}}=\order{n}$和$N_{\text{max}}=2$。
根据式~\eqref{eq:spd:N_M}满足条件的Pauli路径的数量$N_M$最多为:
\begin{equation}
    N_M=\mathrm{Poly}(n) (nLN_{\text{max}})^{M+1}.
\end{equation}
因此,我们可以得到计算$\langle O \rangle ^{(M)}$的总体计算复杂度为:
\begin{equation}
    C=\order{\mathrm{Poly}(n) (2nL)^{M+1}}.
\end{equation}

对于空间复杂度,根据式~\eqref{eq:spd_space},可以表示为$S = S_s+S_f$,其中$S_s$是枚举Pauli路径过程的空间复杂度,$S_f$是计算每个Pauli路径的贡献的空间复杂度。
对于枚举过程的空间复杂度,根据式~\eqref{eq:spd:space_cost}有:
\begin{equation}
    S_{s}=\order{\mathrm{Poly}(n)+\max\{S_{\text{split}},n\}L}=\order{\mathrm{Poly}(n)+nL}.
\end{equation}
对于计算每个Pauli路径的贡献的空间复杂度,根据式~\eqref{eq:space_f}有:
\begin{equation}
    S_f=\order{n}+\order{\poly(n)}+\order{\poly(n)}=\order{\poly(n)}.
\end{equation}


综上所述,对于Clifford扰动量子线路,在给定截断参数$M$的情况下,使用SPD算法计算$\langle O \rangle ^{(M)}$的时间复杂度$C$和空间复杂度$S$分别为:
\begin{equation}\label{eq:cpc:cost}
    \begin{aligned}
        C&=\mathrm{Poly}(n) \order{(2nL)^{M+1}},\\
        S&=\mathrm{Poly}(n)+nL,
    \end{aligned}
\end{equation}
其中$n$为量子比特数,$L$为线路深度。

\section{误差分析}