\chapter{绪论}

\section{研究背景}

量子力学的建立和发展是20世纪人类科学领域最伟大的科学突破之一,它不仅解决了经典物理学在描述微观世界中的失败,而且在许多方面都超越了经典物理学。量子力学的建立不仅推动了物理学的发展,而且对其他学科的发展也产生了深远的影响,许多新兴的领域和学科就此产生,如量子信息科学。

量子信息科学是量子力学和信息科学的交叉领域,它主要研究如何利用量子力学的特性来处理和传输信息。在当今信息时代,信息的处理和传输已经成为人们生活中不可或缺的一部分。
自1940年代,第一台电子计算机诞生以来,计算机技术已经取得了长足的进步,计算机的性能也在不断提高。
英特尔公司创始人戈登·摩尔在1965年提出的摩尔定律,预言单个集成电路上可容纳的晶体管数目每隔18个到24个月就会翻一番。
这一定律不是一个自然定律,而是人们在经验上总结出来的一个规律,但是在过去的几十年中,摩尔定律一直被证明是正确的。但是随着计算机技术的不断发展,集成电路的尺寸越来越小,逐渐接近原子尺度。
在原子尺度下,经典物理学的规律不再适用,量子效应开始显现,这就导致了集成电路的性能难以继续提高,摩尔定律也逐渐失效。

而随着人工智能、大数据、物联网等新兴技术的发展,对计算机性能的需求也越来越高。在这种情况下,传统的计算机技术已经无法满足人们的需求,因此人们开始寻找新的计算机技术。量子计算机作为一种新型的计算机技术,具有很大的发展潜力,并被认为是未来计算机技术的发展方向之一。

在1980年代,Benioff、Manin、Feynman等人提出了量子计算机的概念~\cite{Benioff1980,Benioff1982a,Benioff1982b,Manin1980,Feynman1982}。
最初的动机是希望利用量子力学的特性来模拟量子系统,如果使用经典计算机来模拟量子系统,需要耗费大量的计算资源,而量子计算机可以更加高效地模拟量子系统。后来的研究表明,量子计算机不仅可以模拟量子系统,而且可以解决一些经典计算机难以处理的问题。
1994年,Shor提出了一种基于量子计算机的大整数分解算法~\cite{shor1994algorithms},该算法可以在多项式时间内分解大整数,这对于现代密码学体系具有重要意义。
时至今日,量子计算的研究吸引了众多科学家的关注,也在各个领域衍生出了广泛的应用,包括密码学、量子化学~\cite{peruzzo2014variational, Kandala2017hardware,li2022toward}、量子机器学习~\cite{beer2020training,huang2021experimental,havlivcek2019supervised,mitarai2018quantum}、量子优化算法~\cite{farhi2014quantum,moll2018quantum}等。
此外,除了量子计算之外,量子信息还包括量子纠错、量子复杂度、量子通信、量子传感、量子控制等领域,这些领域也为量子信息科学的发展提供了新的思路和方法。

随着实验技术的进步,量子计算机的实际实现也取得了长足的进步。现在基于各种不同的物理系统,包括离子阱、超导量子比特、量子点、拓扑量子比特等,各种量子计算机的实验平台已经相继建立~\cite{blatt2008quantum,devoret2013superconducting,wallraff2004strong,loss1998quantum}。自2019年以来,研究人员相继在超导量子比特、光量子等平台展示了量子计算在特定任务上对经典计算机的超越,实现了包括随机线路采样~\cite{arute2019quantum}、高斯波色采样等~\cite{zhong2020quantum}等特定任务的量子优越性验证。
近些年来,量子计算机已经发展到了含噪声中等规模量子(Noisy Intermediate-Scale Quantum, NISQ)阶段,这一阶段的量子计算机拥有数十到数百个量子比特,暂无法实现完全容错的量子计算,但是拥有在特定任务上超越经典计算机的潜力。

另一方面,量子计算机的发展也面临着许多困难和挑战。量子计算机的硬件系统非常脆弱,容易受到外界干扰,导致量子比特发生错误。保护量子比特免受外界干扰是量子计算机研究的一个重要课题,量子纠错技术(Quantum Error Correction,QEC)就是为了解决这个问题,通过将量子信息编码在纠错码中,可以有效地保护量子比特免受外界干扰。量子纠错技术是量子计算机研究的一个重要方向,也是实现大规模量子计算的关键技术之一。
噪声阈值定理保证了只要量子比特的错误率低于某个阈值,就可以通过扩大纠错码的规模来实现任意精度的量子计算~\cite{aharonov1996quantum,aliferis2009fault}。
近期,研究人员先后在光量子、超导量子平台上取得了里程碑式的量子纠错实验进展,展示了低于噪声阈值的量子纠错~\cite{reed2012realization,ofek2016extending,riste2015detecting}。
尽管量子计算已经取得了长足的进步,但是要实现大规模量子计算仍然面临着许多困难和挑战,若要真正在生产环境中应用量子计算,还需要解决许多问题,还需要持续在理论和实验方面继续探索和研究。


\section{量子力学简介}
量子力学的建立源于19世纪末20世纪初,当时的物理学家们在研究原子和分子的结构时,发现了一些经典物理学无法解释的现象,如黑体辐射、光电效应、电子的双缝干涉等。20世纪初,普朗克、爱因斯坦、德布罗意、玻尔等人相继提出了量子假设、光量子假设、波粒二象性等概念,奠定了量子力学的基础。
在本节中,我们将简要介绍与量子计算相关的量子力学基础知识,包括量子比特、量子态、量子门、量子纠错等。

\subsection{量子态与密度矩阵}

对于一个纯态量子系统,它在某一时刻的状态可以用一个希尔伯特空间中的一个矢量来表示,称为量子态矢量,利用Dirac符号表示为$|\psi\rangle$。Dirac符号$|\psi\rangle$的数学形式等价于一个N维复线形空间中的列向量,其中N是希尔伯特空间的维度,$|\psi\rangle$称为右矢,代表列向量;$\langle\psi|$称为左矢,代表行向量。两者之间互为共轭转置关系,即$\langle\psi| = |\psi\rangle^\dagger$,其中$\dagger$表示共轭转置。两个量子态之间的内积和外积分别表示为$\langle\psi|\phi\rangle$和$|\psi\rangle\langle\phi|$。

对于一个由$n$个量子比特组成的量子系统,它的状态可以表示为一个$N=2^n$维复数向量,即
\begin{equation}
    |\psi\rangle = \begin{pmatrix} \alpha_0 \\ \alpha_1 \\ \cdots \\ \alpha_{2^n-1} \end{pmatrix} = \sum_{i=0}^{2^n-1} \alpha_i|\phi_i\rangle,
\end{equation}
其中$\alpha_i$是复数,满足$\sum_{i=0}^{2^n-1} |\alpha_i|^2 = 1$,$|\phi_i\rangle$是希尔伯特空间的一组完备的正交归一基矢。
完备性条件$\sum_{i=0}^{2^n-1} |\phi_i\rangle\langle\phi_i| = I$,其中$I$是单位算符。正交性条件$\langle\phi_i|\phi_j\rangle = \delta_{ij}$,其中$\delta_{ij}$是Kronecker delta符号。


对于多个系统构成的复合系统,它的状态可以表示为各个子系统状态的张量积,即
\begin{equation}
    |\psi\rangle = |\psi_1\rangle \otimes |\psi_2\rangle \otimes \cdots \otimes |\psi_n\rangle,
\end{equation}
其中$|\psi_i\rangle$是第$i$个子系统的状态。
因此组合系统中量子态的维度是各个子系统维度的乘积,即$N = N_1 \times N_2 \times \cdots \times N_n$。

对于一个量子系统,它的状态可以是纯态或混合态。纯态是指量子系统处于一个确定的态,可以用上述的矢量来描述;混合态是指量子系统处于一个不确定的态,需要用密度矩阵来描述。密度矩阵是一个厄米矩阵,它可以描述量子系统的统计性质,对于一个纯态量子系统,它的密度矩阵定义为
\begin{equation}
    \rho = |\psi\rangle\langle\psi|,
\end{equation}
其中$|\psi\rangle$是量子系统的状态矢量。对于一个混合态,假设量子系统处于一组纯态$|\psi_i\rangle$的概率为$p_i$,则它的密度矩阵定义为
\begin{equation}
    \rho = \sum_i p_i |\psi_i\rangle\langle\psi_i|,
\end{equation}
其中$p_i$是量子系统处于第$i$个纯态的概率,满足$\sum_i p_i = 1$。
在公理化的量子力学中,密度矩阵满足以下性质:
\begin{enumerate}
    \item 密度矩阵是一个厄米矩阵,即$\rho = \rho^\dagger$;
    \item 密度矩阵是正定的,即对于任意的态矢量$|\psi\rangle$,有$\langle\psi|\rho|\psi\rangle \geq 0$;
    \item 密度矩阵的迹为1,即$\Tr(\rho) = 1$。
\end{enumerate}

\subsection{量子态演化}
量子态的演化是量子计算中的一个重要问题,它描述了量子系统在不同时间点的状态之间的关系。在量子力学中,一个封闭系统的演化是由系统的哈密顿量(Hamiltonian)决定的,t时刻的哈密顿量记为$H(t)$,它描述了系统的动力学性质,满足厄米性质$H(t) = H(t)^\dagger$。孤立系统的随时间的演化可以用薛定谔方程(Schrödinger Equation)来描述,即
\begin{equation}
    i\hbar\frac{\partial}{\partial t}|\psi(t)\rangle = H(t)|\psi(t)\rangle,
\end{equation}
其中$\hbar$是约化普朗克常数,$|\psi(t)\rangle$是系统在时间$t$的状态。薛定谔方程的解可以表示为
\begin{equation}
    |\psi(t)\rangle = U(t)|\psi(0)\rangle,
\end{equation}
其中$U(t) = \exp\left(-\frac{i}{\hbar}\int_0^t H(t')dt'\right)$是时间演化算符,它描述了系统从初始状态$|\psi(0)\rangle$到时间$t$的状态$|\psi(t)\rangle$之间的关系。
时间演化算符$U(t)$是一个幺正算符,满足$U(t)U^\dagger(t) = U^\dagger(t)U(t) = I$。


如果采用密度矩阵来描述量子系统,那么密度矩阵的演化可以用薛定谔方程的密度矩阵形式来描述,即
\begin{equation}
    i\hbar\frac{\partial}{\partial t}\rho(t) = [H(t), \rho(t)],
\end{equation}
其中$[\cdot, \cdot]$表示两个算符的对易子。密度矩阵的演化可以表示为
\begin{equation}
    \rho(t) = U(t)\rho(0)U^\dagger(t).
\end{equation}

\subsection{量子测量}
量子测量是量子力学中的一个重要概念,它描述了量子系统在测量过程中的状态演化。在量子力学中,测量是一个不可逆的过程,它会导致量子系统的状态坍缩。在量子力学中,一个量子系统的可观测量是一个厄米算符$O$,满足$O=\sum_i \lambda_i M_i$,其中$\{M_i\}$是一组半正定厄米算符,满足$M_i \geq 0,M_i=M_i^\dagger$和完备性$\sum_i M_i = I$。因为$M_i \geq 0$,所以存在$\{E_i\}$使得$M_i = E_i^\dagger E_i$,这样的测量算符称为正算子(Positive Operator)。
在量子系统的正算子值测量(Positive Operator-Valued Measure, POVM)过程中,量子系统的状态会坍缩到测量结果对应的态上,即
\begin{equation}
    \rho \rightarrow \frac{E_i\rho E_i^\dagger}{\Tr(E_i\rho E_i^\dagger)},
\end{equation}
得到该状态的概率为$p_i = \Tr(E_i\rho E_i^\dagger)$。在测量过程中,测量结果是随机的,但是测量结果的概率是可以计算的。



通常我们关心的是可观测量的测量期望值,它可以表示为$\langle O \rangle$。期望值可以通过对量子态的多次重复测量得到,制备大量相同的量子态,对每个量子态进行测量,然后对测量结果求平均值。对纯态$\ket{\psi}$的可观测量$O$的期望值可以表示为
\begin{equation}
    \langle O \rangle = \langle \psi|O|\psi\rangle=\sum_i \lambda_i p_i=\sum_i \lambda_i \bra{\psi} E_i^\dagger E_i \ket{\psi},
\end{equation}
其中$p_i = \bra{\psi} E_i^\dagger E_i \ket{\psi}$是测量结果为$\lambda_i$的概率。对于混合态$\rho$,可观测量$O$的期望值可以表示为
\begin{equation}
    \langle O \rangle = \Tr(O\rho)=\sum_i \lambda_i \Tr(E_i\rho E_i^\dagger)=\sum_i \lambda_i \Tr(E_i\rho E_i^\dagger).
\end{equation}

\section{量子计算简介}

量子计算是一种基于量子力学原理的计算模型,它利用量子力学的特性来处理和传输信息。
类比经典计算的基本要素如比特、逻辑门、线路等,量子计算也有类似的基本要素。

\subsection{量子比特}
量子比特(Quantum Bit, Qubit)是量子计算的基本单元,它是量子力学中的最小信息单元,类似于经典计算中的比特(Bit)。
经典比特只能处于0或1两种状态,而量子比特可以处于0和1的叠加态,这通常是对应于二能级量子系统、或是多能级量子系统的两个特定的能级。
为了方便将两个能级记为$|0\rangle$和$|1\rangle$。在矩阵表示中,$|0\rangle$和$|1\rangle$分别表示为
\begin{equation}
    |0\rangle = \begin{pmatrix} 1 \\ 0 \end{pmatrix}, \quad |1\rangle = \begin{pmatrix} 0 \\ 1 \end{pmatrix}.
\end{equation}

一个任意量子比特的状态可以表示为
\begin{equation}
    |\psi\rangle = e^{i\gamma}(\cos\frac{\theta}{2}|0\rangle + e^{i\phi}\sin\frac{\theta}{2}|1\rangle)=e^{i\gamma}\begin{pmatrix}
        \cos\frac{\theta}{2} \\ e^{i\phi}\sin\frac{\theta}{2}
    \end{pmatrix},
\end{equation}
其中$\theta$和$\phi$是两个角度,$\gamma$是一个全局相位,满足$0 \leq \theta \leq \pi$,$0 \leq \phi < 2\pi$,$0 \leq \gamma < 2\pi$。 量子比特的状态可以用Bloch球来表示,Bloch球是一个单位球,它的表面上的点对应于量子比特的状态。

一个由n个量子比特组成的量子系统有$2^n$个本征态,它的状态可以用长度为n的二进制串来表示,即$|x_1x_2\cdots x_n\rangle$,其中$x_i \in \{0, 1\}$。一个n量子比特的量子系统的状态可以表示为
\begin{equation}
    |\psi\rangle = \sum_{x=0}^{2^n-1} \alpha_x |x_1x_2\cdots x_n\rangle,
\end{equation}
其中$\alpha_x$是复数,满足$\sum_{x=0}^{2^n-1} |\alpha_x|^2 = 1$。
有时也会使用十进制表示,即$|x\rangle$,其中$x \in \{0, 1, \cdots, 2^n-1\}$。这样的基矢称为计算基矢(Computational Basis)。

\subsection{量子线路}
量子线路是量子计算中的一个重要概念,它描述了量子比特之间的相互作用关系。量子线路由量子门(Quantum Gate)组成,量子门是用来操作量子比特的基本单元,类似于经典计算中的逻辑门。量子门是一个幺正算符,它作用在量子比特上,改变量子比特的状态。

最简单的一类门是单比特量子门,