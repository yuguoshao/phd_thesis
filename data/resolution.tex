% !TeX root = ../thuthesis-example.tex

\begin{resolution}
  经典模拟是研究量子算法、量子纠错、校准量子计算机的重要工具。现如今随着量子硬件的发展,传统的经典模拟方法已经逐渐逼近性能极限。在当前的时代背景下,变分量子算法因为不需要昂贵的纠错协议受到了广泛关注,但变分量子算法能否在线路噪声等实际环境下实现量子优势仍然是个困难的、被反复讨论的开放问题。

邵钰菓的博士学位论文《基于Pauli路径积分模拟量子线路》聚焦NISQ时代量子线路的经典模拟难题,提出了基于Pauli路径积分的泡利基下反向路径积分(OBPPP)算法并拓展了稀疏泡利动力学(SPD)算法,系统分析了算法的计算复杂度与误差特性,揭示了噪声对量子优势边界的临界影响。论文的主要成果如下:


1. 证明了绝大多数含噪声变分量子线路观测期望值在一定条件下的经典可模拟性。

2. 证明了在Pauli噪声率恒定或高于1/logL时,观测期望值可被多项式复杂度经典模拟;而当噪声率低于1/L时,模拟复杂度可能指数增长。

3. 通过复现IBM 127量子比特Eagle处理器的Ising模型实验,验证了该模拟算法在精度与效率上显著优于实际量子设备,并实现了噪声强度的量化评估。

研究兼具理论深度与实践价值,为量子硬件验证及噪声抑制提供了重要工具。


论文选题前沿,逻辑严谨,成果发表于《Physical Review Letters》等高水平期刊,创新性突出,理论意义与应用价值显著。答辩过程中,邵钰菓陈述清晰,表达严谨,准确回答了委员会提出的问题,展现了扎实的专业功底与独立科研能力。


经答辩委员会讨论与无记名投票,一致通过邵钰菓的博士学位论文答辩,认为这是一篇优秀的博士学位论文,建议授予邵钰菓理学博士学位。


\end{resolution}
