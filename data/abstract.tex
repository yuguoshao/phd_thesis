% !TeX root = ../thuthesis-example.tex

% 中英文摘要和关键字

\begin{abstract}
  大规模变分量子算法被广泛认为是实现实用量子优势的潜在途径。然而,量子噪声的存在可能会抑制并削弱这些优势,这使得经典可模拟性的界限变得模糊。为了进一步澄清这一问题,我们提出了一种基于可观测量在泡利路径上反向传播路径积分(OBPPP)的新型多项式规模方法。该方法能够在单量子比特泡利噪声存在的情况下,以有界截断误差高效近似变分量子算法中算子的期望值。理论上我们严格证明:1)当最小非零噪声率γ为常数时,OBPPP的时间与空间复杂度与量子比特数n、电路深度L呈现多项式关系;2)对于可变的γ,当存在超过两个非零噪声因子时,若γ大于1/log L则复杂度保持Poly(n, L),但若γ低于1/L时复杂度将随L呈指数增长。数值方面,我们对IBM在127量子比特鹰处理器上进行的零噪声外推实验结果[Nature 618, 500 (2023)]进行了经典模拟。相比量子设备,我们的方法获得了更高的精度与更快的运行速度。此外,我们的方法能够模拟含噪声结果,精确复现了IBM未经误差缓解、与原始实验观测直接对应的计算结果。本研究表明噪声在经典模拟中的关键作用,所提出的方法适用于广泛量子电路的期望值计算,在量子计算机验证领域具有普适性。


  论文的摘要是对论文研究内容和成果的高度概括。
  摘要应对论文所研究的问题及其研究目的进行描述,对研究方法和过程进行简单介绍,对研究成果和所得结论进行概括。
  摘要应具有独立性和自明性,其内容应包含与论文全文同等量的主要信息。
  使读者即使不阅读全文,通过摘要就能了解论文的总体内容和主要成果。

  论文摘要的书写应力求精确、简明。
  切忌写成对论文书写内容进行提要的形式,尤其要避免“第 1 章……;第 2 章……;……”这种或类似的陈述方式。

  关键词是为了文献标引工作、用以表示全文主要内容信息的单词或术语。
  关键词不超过 5 个,每个关键词中间用分号分隔。

  % 关键词用“英文逗号”分隔,输出时会自动处理为正确的分隔符
  \thusetup{
    keywords = {量子计算, 量子线路经典模拟 , Pauli路径积分, 量子算法, 量子噪声},
  }
\end{abstract}

\begin{abstract*}
  An abstract of a dissertation is a summary and extraction of research work and contributions.
  Included in an abstract should be description of research topic and research objective, brief introduction to methodology and research process, and summary of conclusion and contributions of the research.
  An abstract should be characterized by independence and clarity and carry identical information with the dissertation.
  It should be such that the general idea and major contributions of the dissertation are conveyed without reading the dissertation.

  An abstract should be concise and to the point.
  It is a misunderstanding to make an abstract an outline of the dissertation and words “the first chapter”, “the second chapter” and the like should be avoided in the abstract.

  Keywords are terms used in a dissertation for indexing, reflecting core information of the dissertation.
  An abstract may contain a maximum of 5 keywords, with semi-colons used in between to separate one another.

  % Use comma as separator when inputting
  \thusetup{
    keywords* = {keyword 1, keyword 2, keyword 3, keyword 4, keyword 5},
  }
\end{abstract*}
