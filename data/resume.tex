% !TeX root = ../thuthesis-example.tex

\begin{resume}

  \section*{个人简历}

  1997 年 11 月 11 日出生于四川省自贡市。

  2016 年 9 月考入华中科技大学数学与统计学院数学与应用数学专业,2020 年 6 月本科毕业并获得理学学士学位。

  2020 年 9 月免试进入清华大学数学科学系攻读数学博士至今。


  \section*{在学期间完成的相关学术成果}

  \subsection*{学术论文}

  \begin{achievements}
    \item Shao, Y., Wei, F., Cheng, S., \& Liu, Z. (2024). Simulating noisy variational quantum algorithms: A polynomial approach. Physical Review Letters, 133(12), 120603.
    \item Sun, W., Wei, F., Shao, Y., \& Wei, Z. (2024). Sudden death of quantum advantage in correlation generations. Science Advances, 10(47), eadr5002.
    \item Huang, Y., Shao, Y., Ren, W., Sun, J., \& Lv, D. (2023). Efficient quantum imaginary time evolution by drifting real-time evolution: An approach with low gate and measurement complexity. Journal of Chemical Theory and Computation, 19(13), 3868-3876.
    \item Shao, Y., Li, Y., Wei, F., Zhan, H., Wang, B., Wei, Z., Zhang, L., \& Liu, Z. (2024). Variational Graphical Quantum Error Correction Codes: adjustable codes from topological insights. arXiv preprint arXiv:2410.02608.
    \item Zhang, R., Shao, Y., Wei, F., Cheng, S., Wei, Z., \& Liu, Z. (2024). Clifford Perturbation Approximation for Quantum Error Mitigation. arXiv preprint arXiv:2412.09518.
  \end{achievements}

  %\subsection*{专利}

  %\begin{achievements}
    %\item 任天令, 杨轶, 朱一平, 等. 硅基铁电微声学传感器畴极化区域控制和电极连接的方法: 中国, CN1602118A[P]. 2005-03-30.
    %\item Ren T L, Yang Y, Zhu Y P, et al. Piezoelectric micro acoustic sensor based on ferroelectric materials: USA, No.11/215, 102[P]. (美国发明专利申请号.)
  %\end{achievements}

\end{resume}
