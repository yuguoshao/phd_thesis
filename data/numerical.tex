\chapter{数值模拟}

在过去的章节中,我们介绍了量子线路的基本概念,并介绍了如何使用基于Pauli路径积分的模拟算法来模拟量子线路。并在一些具体的线路模型中,研究了模拟算法的计算复杂度和误差。特别地,发现了在一些情况下,比如含噪声的变分量子线路,算法的计算复杂度关于线路规模是多项式关系。
虽然,我们定性地分析了模拟算法的计算复杂度,但是我们还没有给出具体的数值模拟结果。多项式复杂度对于实际的模拟算法来说是一个很好的性质,但是是否实用还有待进一步的数值模拟结果的验证。
例如,在定理~\ref{thm:main}中,我们得到了对于噪声\textbf{情况1}时,算法的计算复杂度是:
\begin{equation*}
    \mathrm{Poly}(n) \order{L} \bigg(\frac{\norm{O}_\infty}{\varepsilon \sqrt{\delta}} \bigg)^{\order{\frac{1}{\gamma}}},
\end{equation*}
其中$n$是线路的规模,$L$是线路的深度,$\varepsilon$是算法的精度,$\delta$是算法失败的概率,$\gamma=\min\{p|{p \in \{p_x,p_y,p_z\},p\neq 0}\}$是噪声率,$\norm{O}_\infty$是可观测量的无穷范数。当噪声强度$\gamma$很小时,例如$\gamma=0.01$,我们可以得到算法的计算复杂度是$\mathrm{Poly}(n) \order{L} \bigg(\frac{\norm{O}_\infty}{\varepsilon \sqrt{\delta}} \bigg)^{{100}}$,这是一个难以实现的计算复杂度。万幸的是,实际情况下的计算复杂度并不会达到这么高的程度,我们将在本章中具体说明。

除此之外,在这一章中,我们将使用基于Pauli路径积分的模拟算法来模拟一些量子线路的具体实例。并与实际的实验结果进行比较,以验证模拟算法的有效性。

在本章中,如未加说明,我们将假设环境噪声处于\textbf{情况1},这是因为在实际的量子计算机中,很难出现纯净的单独类型的Pauli错误。因此假设$p_x\neq 0,p_y\neq 0,p_z\neq 0$是合理的,此时与噪声关联的Hamming Weight 根据式~\eqref{eq:noise_hamming_weight}可以得到:
\begin{equation*}
    \abs{\bm{s}}_{\mathcal{N}}=\abs{\bm{s}}_X + \abs{\bm{s}}_Y + \abs{\bm{s}}_Z.
\end{equation*}
在这种情况下,正是原始的Hamming Weight定义,为了方便,我们将在本章中使用$\abs{\bm{s}}$来表示。


\section{模拟算法的计算复杂度分析}
对于变分量子线路,根据式~\eqref{eq:vqa:cost},在噪声\textbf{情况1}下,模拟算法的计算复杂度是:
\begin{equation*}
    \mathrm{Poly}(n) \order{L}2^M,
\end{equation*}
其中$n$是线路的规模,$L$是线路的深度,$M$是截断变量。可以看到在过去理论估计中,计算复杂度关于截断变量$M$是指数关系。这个指数因子的来源是由于在Pauli 路径枚举过程中,根据式~\eqref{eq:vqa:case1:N_M},符合条件$\abs{\bm{s}}\leq M$的Pauli路径数目是指数级别的。在这一节中,我们将通过具体的数值模拟结果来估计,给定截断变量$M$时,满足条件Pauli路径的真实数目。


对于截断变量的取值,根据引理~\ref{lemma:MSE_l},我们可以得到
对于变分量子线路,如果式~\eqref{eq:E_cross_equals_0}成立,那么对任意的$\nu > 0$,只要截断参数$M$满足:
\begin{equation}
    M\geq\frac{1}{4\gamma}\ln{\frac{\norm{O}_\infty^2}{\nu}},
\end{equation}
其中$\gamma:=\min\{p|{p \in \{p_x,p_y,p_z\},p\neq 0}\}$,那么含噪声期望值模拟的均方误差满足:
\begin{equation}
    \mathbb{E}_{\bm{\theta}}\left[\left(\widetilde{\langle O\rangle}-\widehat{\langle O\rangle}\right)^2\right]\leq\nu.
\end{equation}
事实上,在实际计算中,$M$也并不需要取到这么大的值,我们将在本节中具体用数值说明。

\subsection{计算复杂度与截断变量$M$的关系}




\section{Ising模型演化}