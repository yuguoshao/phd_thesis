% !TeX root = ../thuthesis-example.tex

\begin{comments}
% \begin{comments}[name = {指导小组评语}]
% \begin{comments}[name = {Comments from Thesis Supervisor}]
% \begin{comments}[name = {Comments from Thesis Supervision Committee}]

  邵钰菓同学完成了一流的博士毕业论文。

目前的量子计算处于所谓的含噪声中等规模量子硬件时代(NISQ era)。在这个时代背景下,变分量子算法凭借其对量子硬件更低的要求,有望率先实现实际可用和有意义的量子算法,使得其在组合优化,量子化学,材料计算乃至机器学习等领域都备受关注。但变分量子算法的现状面临着诸多争议:线路噪声带来的退相干,表达能力的局限,可训练性的丧失。事实上,找到相对经典算法真正具备量子优势的变分量子算法依旧是一个被反复讨论又困难重重的开放问题。那么,一个同样重要的反问题是:可否指出什么类型的变分量子算法是容易被经典模拟的?

在论文中,邵钰菓同学提出理论证明了大部分含噪声变分量子线路算符期望的经典可模拟性,从而否定这些变分量子算法的量子优越性,并根据此理论构造出了一个切实可算的多项式复杂度的经典算法——泡利基下反向路径积分(OBPPP)。相关成果以“Simulating Noisy Variational Quantum Algorithms: A Polynomial Approach”为题发表于《物理学评论快报》(Physical Review Letters).

在数值方面成功地对 IBM 的 Eagle 量子处理器上的变分量子算法进行了经典模拟,运行时间比量子硬件更短,同时实现了更准确的期望。以IBM这个实验为例,比特数为127, 线路深度为80,OBPPP得到每张图的计算时间均小于5分钟(基于2张Xeon 6330 CPU)。此外,对不同的噪声率,论文的方法还可以根据路径压制因子很容易地推出对应算符期望值,从而能与未修正的原始实验数据做直接对比,并表现出与实验结果很强的一致性。

论文反映出邵钰菓对含噪声量子电路的经典模拟的深刻见解。研究成果具有很高的原创性和重要的学术价值,对于研究量子优越性有重要指导意义。这一工作更细致地揭示了噪声与经典可模拟性的关系,同时也展示了该方法在更广泛类别的量子电路中的适用性。


\end{comments}
