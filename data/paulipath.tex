\chapter{Pauli路径积分模拟}



\section{量子计算的经典模拟}
随着量子计算硬件的快速发展,量子比特规模已达到数百量级(如IBM Quantum Heron处理器)。然而,当前含噪声中等规模量子(NISQ)设备的计算保真度仍受限于退相干时间和门操作误差。在此背景下,量子计算的经典模拟技术具有双重意义:
\begin{itemize}
    \item 量子计算的经典模拟是验证量子计算硬件的有效手段。通过经典模拟,我们可以验证量子计算硬件的正确性,评估其性能,甚至优化量子算法。
    \item 突破现有量子计算硬件的规模限制。通过经典模拟,我们可以模拟更大规模的量子系统,以探索量子计算的潜在应用(如数千量子比特)。
    \item 辅助算法设计。通过经典模拟,为算法提供参数优化、误差分析等支持。
\end{itemize}
常见的量子计算经典模拟方法包括全状态模拟,全状态模拟全状态模拟(Full-State Simulation)是量子计算经典模拟的一种直接方法,即在经典计算机上存储和演化完整的量子态向量或密度矩阵。

在纯态量子计算模型中,一个$n$量子比特的量子态可以用$2^n$维复向量表示:
\begin{equation}
    |\psi\rangle = \sum_{i=0}^{2^n-1} c_i |i\rangle,
    \quad c_i \in \mathbb{C}, \quad \sum_{i=0}^{2^n-1} |c_i|^2 = 1.
\end{equation}
在态向量模拟中,整个量子态向量被存储在计算机内存中。并在应用量子门时,通过矩阵乘法和线性代数运算来模拟量子态的演化:
\begin{equation}
    |\psi'\rangle = U |\psi\rangle,
\end{equation}
其中$U$是一个$2^n \times 2^n$的酉矩阵,表示量子门的作用。
在计算复杂度上,应用一个量子门的计算复杂度为$O(2^n)$。对于存储复杂度,一个$n$量子比特的量子态需要$2^n$个复数来存储。

对于模拟含噪声的量子计算硬件,或是模拟包含